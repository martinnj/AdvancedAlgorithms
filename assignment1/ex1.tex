\section{Exercise 1: $b$-flow}
A flow is a b-flow if its satisfies the following
\begin{align}
\sum_{e \in \delta^- (v)} x_e - \sum_{e \in \delta^+ (v)} x_e &= b_v, \forall v \in V \label{eq:1}\\
0 \leq x_e &\leq u_e, \forall e \in E \label{eq:2}
\end{align}

\noindent Below we have illustrated the $b$-flows for figure (a). We see that
each node satisfies equation \ref{eq:1} and \ref{eq:2}, giving us the $b$-flows
\begin{align*}
v_2 v_4 &= 2\\
v_5 v_1 &= 3\\
v_5 v_3 &= 4
\end{align*} 
which is illustrated in figure 2. 

\begin{minipage}{.5\textwidth}
\centering
\begin{tikzpicture}[->,>=stealth',shorten >=1pt,auto,node distance=3cm,
                    semithick]
  \tikzstyle{every state}=[fill=white,text=black]

  \node[state,label=150:$3$] (1) {$v_1$};
  \node[state,label=30:$-2$] (2) [right of=1] {$v_2$};
  \node[state,label=150:$4$] (3) [below of=1] {$v_3$};
  \node[state,label=50:$2$] (4) [below of=2] {$v_4$};
  \node[state,label=93:$-7$] (5) [below right of=3] {$v_5$};

  \path 	(1) 	edge node {1} (4)
  			edge node {4} (3)
		(2)	edge node {5} (1)
			edge [bend left=75] node {3} (5)
			edge node {2} (4)
		(3)	edge node {4} (2)
			edge node {3} (4)
		(4)	edge node {2} (5)
		(5)	edge node {7} (3)
			edge [bend left=75] node {6} (1);
\end{tikzpicture}
\captionof{figure}{(a)}
\end{minipage}%
\begin{minipage}{.5\textwidth}
\centering
\begin{tikzpicture}[->,>=stealth',shorten >=1pt,auto,node distance=3cm,
                    semithick]
  \tikzstyle{every state}=[fill=white,text=black]

  \node[state,label=150:$3$] (1) {$v_1$};
  \node[state,label=30:$-2$] (2) [right of=1] {$v_2$};
  \node[state,label=50:$4$] (3) [below of=1] {$v_3$};
  \node[state,label=50:$2$] (4) [below of=2] {$v_4$};
  \node[state,label=93:$-7$] (5) [below right of=3] {$v_5$};

  \path 	(2)	edge node {2} (4)
		(5)	edge node {4} (3)
			edge [bend left=75] node {3} (1);
\end{tikzpicture}
\captionof{figure}{(a) $b$-flow}
\end{minipage}

\noindent In figure (b) we can only satisfy equation \ref{eq:1} and \ref{eq:2}
with some of the nodes. Due to the fact that vertex $v_4$ has no outgoing edges 
and we do not allow negative flows, we can not fulfil the demand of -2. Because of 
that, we have no $b$-flow here.

\begin{minipage}{.5\textwidth}
\centering
\begin{tikzpicture}[->,>=stealth',shorten >=1pt,auto,node distance=3cm,
                    semithick]
  \tikzstyle{every state}=[fill=white,text=black]

  \node[state,label=150:$3$] (1) {$v_1$};
  \node[state,label=30:$1$] (2) [right of=1] {$v_2$};
  \node[state,label=-50:$-2$] (3) [below of=1] {$v_3$};
  \node[state,label=-50:$-2$] (4) [below of=2] {$v_4$};

  \path 	(1) 	edge node {4} (3)
		(2)	edge node {3} (1)
			edge node {6} (4)
		(3)	edge node {2} (2)
			edge node {3} (4);
\end{tikzpicture}
\captionof{figure}{(b)}
\end{minipage}%
\begin{minipage}{.5\textwidth}
\centering
\begin{tikzpicture}[->,>=stealth',shorten >=1pt,auto,node distance=3cm,
                    semithick]
  \tikzstyle{every state}=[fill=white,text=black]

  \node[state,label=150:$3$] (1) {$v_1$};
  \node[state,label=30:$1$] (2) [right of=1] {$v_2$};
  \node[state,label=-50:$-2$] (3) [below of=1] {$v_3$};
  \node[state,label=-50:$-2$] (4) [below of=2] {$v_4$};

  \end{tikzpicture}
\captionof{figure}{(b) $b$-flow}
\end{minipage}