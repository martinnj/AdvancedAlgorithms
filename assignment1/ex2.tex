\section{Exercise 2: An application of MCFP: rectilinear planar embedding}

\subsection{Exercise 2.1}
\todo[inline]{Table description should be better. Also the reference can't find
  the correct number.}

The $x_{vf}$ values for all vertices and and faces of Figure
3 in \cite{assignment1} can be found in Table \ref{table:xvf}.

\begin{table}[h]
\centering
\begin{tabular}{l|lllll}
$x_{vf}$ & $a$ & $b$ & $c$ & $d$ & $e$ \\ \hline
$v_1$   & 0   & 1   & 1   & 0   & 0   \\
$v_2$   & 0   & 0   & 1   & 1   & 0   \\
$v_3$   & 1   & 0   & 1   & 1   & 1   \\
$v_4$   & 0   & 0   & 0   & -1  & 1   \\
$v_5$   & 1   & 0   & 0   & 0   & -1  \\
$v_6$   & 1   & 1   & 0   & 1   & 1   \\
$v_7$   & 0   & 0   & 0   & 0   & 0
\end{tabular}
\label{table:xvf}
\caption{$x_{vf}$-values for all vertex/face combinations.}
\end{table}

The $z_{fg}$ values for the same graph can be found in Table \ref{table:zfg}.

\begin{table}[h]
\centering
\begin{tabular}{l|lllll}
$z_{fg}$ & $a$ & $b$ & $c$ & $d$ & $e$ \\ \hline
$a$     &     & 0   & 1   &     & 0   \\
$b$     & 2   &     & 1   & 1   &     \\
$c$     & 1   & 1   &     & 0   &     \\
$d$     &     & 1   & 0   &     & 2   \\
$e$     & 4   &     &     & 0   &
\end{tabular}
\label{table:zfg}
\caption{$z_{fg}$ values for all faces, values for empty sets are not displayed}
\end{table}

Furthermore, there is a total of $12$ breakpoints in the graph.

A drawing of a rectilinear layout for Figure 3 in \cite{assignment1} can be seen
in Figure \ref{fig:rect}.

\begin{tikzpicture}[-,>=stealth',shorten >=1pt,auto,node distance=1.3cm,
                    semithick]
  \tikzstyle{every state}=[fill=white,text=black]

  \node[state] (1) {$v_1$};
  \node[state] (6) [below of=1]{$v_6$};
  \node[state] (2) [left of=6]{$v_2$};
  \node[state] (3) [below of=2]{$v_3$};
  \node[state] (4) [right of=6]{$v_4$};
  \node[state] (5) [left of=2]{$v_5$};

\path (1) edge (6)
      (1) edge (2)
      (1) edge (5)
      (1) edge (4)

      (2) edge (5)
      (2) edge (6)

      (3) edge (5)
      (3) edge (2)
      (3) edge (6)
      (3) edge (4)

      (4) edge (5)

      (6) edge (4);
\end{tikzpicture}

\subsection{Exercise 2.2}
\subsection{Exercise 2.3}
\subsection{Exercise 2.4}
\subsection{Exercise 2.5}