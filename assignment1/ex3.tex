\section{Exercise 3: Reduction to MCFP}
%Unfortunately we did not manage to break this exercise, even though we have
%tried for several hours.  If we could get some help with it, we would appreciate
%it very much, and we would be happy to solve it for the re-handin.

In this assignment there is 4 cases that should be taken into account with
regards to the $l$ and $u$ capacities. These cases can be seen in Figure \ref{fig:cases}.

\begin{figure}[H]
\centering

\begin{minipage}{.22\textwidth}
\centering
\begin{tikzpicture}[->,>=stealth',shorten >=1pt,auto,node distance=3cm,
                    semithick]
  \tikzstyle{every state}=[fill=white,text=black]

  \node[state] (1) {$u$};
  \node[state] (2) [below of=1] {$v$};

  \path 	(1) 	edge node[align=center] {$k$\\$l$} (2);
\end{tikzpicture}
\captionof{subfigure}{Case where both $u_e$ and $l_e$ are finite constants.}
\label{fig:case1}
\end{minipage}%
\quad
\begin{minipage}{.22\textwidth}
\centering
\begin{tikzpicture}[->,>=stealth',shorten >=1pt,auto,node distance=3cm,
                    semithick]
  \tikzstyle{every state}=[fill=white,text=black]

  \node[state] (1) {$u$};
  \node[state] (2) [below of=1] {$v$};

  \path 	(1) 	edge node[align=center] {$k$\\$-\infty$} (2);
\end{tikzpicture}
\captionof{subfigure}{Case where both $u_e$ are a finite constant and $l_e$ is
  negative infinite.}
\label{fig:case2}
\end{minipage}%
\quad
\begin{minipage}{.22\textwidth}
\centering
\begin{tikzpicture}[->,>=stealth',shorten >=1pt,auto,node distance=3cm,
                    semithick]
  \tikzstyle{every state}=[fill=white,text=black]

  \node[state] (1) {$u$};
  \node[state] (2) [below of=1] {$v$};

  \path 	(1) 	edge node[align=center] {$\infty$\\$l$} (2);
\end{tikzpicture}
\captionof{subfigure}{Case where $u_e$ is infinite and $l_e$ is a constant.}
\label{fig:case3}
\end{minipage}%
\quad
\begin{minipage}{.22\textwidth}
\centering
\begin{tikzpicture}[->,>=stealth',shorten >=1pt,auto,node distance=3cm,
                    semithick]
  \tikzstyle{every state}=[fill=white,text=black]

  \node[state] (1) {$u$};
  \node[state] (2) [below of=1] {$v$};

  \path 	(1) 	edge node[align=center] {$\infty$\\$-\infty$} (2);
\end{tikzpicture}
\captionof{subfigure}{Case where $u_e$ are infinite and $l_e$ is negative
  infinite.}
\label{fig:case4}
\end{minipage}%


\caption{The different cases for vertices and edges.}
\label{fig:cases}
\end{figure}

\subsection{Exercise 3.1}
The only case we need to handle here is case 4, as seen in Figure
\ref{fig:case4}.  If we have a graph $G=(V,E)$ where $V=\{u,v\}$ and
$E=\{(u,v)\}$, with $l_{(u,v)} = -\infty$ and $u_{(u,v)}=\infty$ we can view the
negative $l$ capacity as an edges ability to carry flow in the reverse
direction. But inserting an anti-parallel edge $(v,u)$ with $u_{(v,u)} =
\infty$, this will make the $l$ value for both edges equal to 0. Figure
\ref{fig:3_1_ex1} shows the example graph before and after this operation.

\begin{figure}[H]
\centering
\begin{minipage}{.25\textwidth}
\centering
\begin{tikzpicture}[->,>=stealth',shorten >=1pt,auto,node distance=3cm,
                    semithick]
  \tikzstyle{every state}=[fill=white,text=black]

  \node[state] (1) {$u$};
  \node[state] (2) [below of=1] {$v$};

  \path 	(1) 	edge node[align=center] {$\infty$\\$-\infty$} (2);
\end{tikzpicture}
\captionof{subfigure}{The initial graph $G$ with two vertices and on edge.}
\label{fig:3_1_ex1a}
\end{minipage}%
$\Longrightarrow$
\begin{minipage}{.25\textwidth}
\centering
\begin{tikzpicture}[->,>=stealth',shorten >=1pt,auto,node distance=3cm,
                    semithick]
  \tikzstyle{every state}=[fill=white,text=black]


  \node[state] (1) {$u$};
  \node[state] (2) [below of=1] {$v$};

  \path 	(1) 	edge[bend left=25] node[align=center] {$\infty$\\$0$} (2)
		(2)	edge[bend left=25] node[align=center] {$\infty$\\$0$} (1);
\end{tikzpicture}
\captionof{subfigure}{The resulting graph $G$ with 2 vertices and 2 edges.}
\label{fig:3_1_ex1b}
\end{minipage}

\caption{The result of the first part of the operation.}
\label{fig:3_1_ex1}
\end{figure}

Since we cannot have anti-parallel edges we will insert an extra vertex and
connect one of the edges to this vertex and add a new edge. The new vertex $w$
shall have a demand $b = 0$ so as to not consume or produce any additional
flow. The edge $(v,u)$ shall be changed to $(v,w)$ and a new edge with $(w,u)$
shall be introduced with the same capacities but with cost $c(w,u) = 0$. Figure
\ref{fig:3_1_ex2} Illustrates this example.

\begin{figure}[H]
\centering
\begin{minipage}{.25\textwidth}
\centering
\begin{tikzpicture}[->,>=stealth',shorten >=1pt,auto,node distance=3cm,
                    semithick]
  \tikzstyle{every state}=[fill=white,text=black]


  \node[state] (1) {$u$};
  \node[state] (2) [below of=1] {$v$};

  \path 	(1) 	edge[bend left=25] node[align=center] {$\infty$\\$0$} (2)
		(2)	edge[bend left=25] node[align=center] {$\infty$\\$0$} (1);
\end{tikzpicture}
\captionof{subfigure}{The graph $G$ after the first operation.}
\label{fig:3_1_ex2a}
\end{minipage}%
$\Longrightarrow$
\begin{minipage}{.25\textwidth}
\centering
\begin{tikzpicture}[->,>=stealth',shorten >=1pt,auto,node distance=3cm,
                    semithick]
  \tikzstyle{every state}=[fill=white,text=black]


  \node[state] (1) {$u$};
  \node[state] (2) [below of=1] {$v$};
  \node[state] (3) [below left of=1] {$w$};

  \path 	(1) 	edge[bend left=25] node[align=center] {$\infty$\\$0$} (2)
		(2)	edge[bend left=25] node[align=center] {$\infty$\\$0$} (3)
                (3)     edge[bend left=25] node[align=center] {$\infty$\\$0$} (1);
\end{tikzpicture}
\captionof{subfigure}{The final graph $G$ at the end of the procedure.}
\label{fig:3_1_ex2b}
\end{minipage}

\caption{The result of the entire operation.}
\label{fig:3_1_ex2}
\end{figure}

\subsection{Exercise 3.2}

For this part we need only consider case 2 as seen in Figure \ref{fig:case2}
since we have already solved this for case 4 in the previous question.  This
problem is solvable using the exact same method as above, only the $u_e$ values
are different. The reduction can be seen in Figure \ref{fig:3_2_ex1}

\begin{figure}[H]
\centering
\begin{minipage}{.25\textwidth}
\centering
\begin{tikzpicture}[->,>=stealth',shorten >=1pt,auto,node distance=3cm,
                    semithick]
  \tikzstyle{every state}=[fill=white,text=black]

  \node[state] (1) {$u$};
  \node[state] (2) [below of=1] {$v$};

  \path 	(1) 	edge node[align=center] {$k$\\$-\infty$} (2);
\end{tikzpicture}
\captionof{subfigure}{The graph $G$ in the beginning.}
\label{fig:3_2_ex1a}
\end{minipage}%
$\Longrightarrow$
\begin{minipage}{.25\textwidth}
\centering
\begin{tikzpicture}[->,>=stealth',shorten >=1pt,auto,node distance=3cm,
                    semithick]
  \tikzstyle{every state}=[fill=white,text=black]


  \node[state] (1) {$u$};
  \node[state] (2) [below of=1] {$v$};
  \node[state] (3) [below left of=1] {$w$};

  \path 	(1) 	edge[bend left=25] node[align=center] {$k$\\$0$} (2)
		(2)	edge[bend left=25] node[align=center] {$k$\\$0$} (3)
                (3)     edge[bend left=25] node[align=center] {$k$\\$0$} (1);
\end{tikzpicture}
\captionof{subfigure}{The final graph $G$ at the end of the procedure.}
\label{fig:3_2_ex1b}
\end{minipage}

\caption{The result of the entire operation.}
\label{fig:3_2_ex1}
\end{figure}

\subsection{Exercise 3.3}

In the two previous questions we did this for case 2 and 4 (Figure
\ref{fig:case2} and \ref{fig:case4}). We will now pay attention to case 1 and 3
(Figure \ref{fig:case1} and \ref{fig:case3}.) It is worth noticing that with
these two casses there is no requirement that the finite constant $l$ be
negative, it might also be positive.

\subsection{Exercise 3.4}
\subsection{Exercise 3.5 (Optional)}
