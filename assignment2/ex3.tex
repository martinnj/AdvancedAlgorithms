\section{Bottom-$k$ sampling with strong universality}
\subsection{Exercise 5}

\subsection{Exercise 6}
% Expansion of the formula, for clarification.
% \begin{align*}
%   \text{Pr}&\left[ |X - \mu| \geq r\sqrt{\mu} \right] &\leq \frac{1}{r^2} \\
%   \text{Pr}&\left[ ||S_{h,p}(A)| - \mu| \geq r\sqrt{\mu} \right] &\leq \frac{1}{r^2} \\
%   \text{Pr}&\left[ ||S_{h,p}(A)| - \text{E}\left[X\right]| \geq r\sqrt{\text{E}\left[X\right]} \right] &\leq \frac{1}{r^2} \\
%   \text{Pr}&\left[ ||S_{h,p}(A)| - \text{E}\left[|S_{h,p}(A)|\right]| \geq r\sqrt{\text{E}\left[|S_{h,p}(A)|\right]} \right] &\leq \frac{1}{r^2} \\
% \end{align*}

(I) The number of elements from A that hash below p is less than k.\\
For $r \geq 1$ it holds trivially. So we need to prove it for $0 < r \leq 1$.
\begin{align*}
  0 < r \leq \frac{\sqrt{k}}{3}\\
  P_{(I)} &= \text{Pr}\left[ X_A < k \right] &\leq \frac{1}{r^2} \\
  P_{(I)} &= \text{Pr}\left[ \mu_A(1-r/\sqrt{k}) \right] &\leq \frac{1}{r^2} \\
  P_{(I)} &= \text{Pr}\left[ E(X_A)(1-r/\sqrt{k}) \right] &\leq \frac{1}{r^2} \\
\end{align*}
Lemma 1:
\begin{align*}
  \text{Pr}&\left[ |X - \mu| \geq r\sqrt{\mu} \right] \leq \frac{1}{r^2} \\
  \text{Pr}&\left[ |X - E(X)| \geq r\sqrt{E(X)} \right] \leq \frac{1}{r^2} \\
  %P_{(I)} &= \text{Pr}\left[ \mu_A(1-r/\sqrt{k}) \right] &\leq \frac{1}{r^2}\\
  %P_{(I)} &= \text{Pr}\left[ |S_{h,p}(A)| < k \right] &\leq \frac{1}{r^2}
\end{align*}


\subsection{Exercise 7}
