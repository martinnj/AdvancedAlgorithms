\section{Bottom-$k$ sampling with strong universality}
\subsection{Exercise 5}
For positive parameters $a < 1$ and $b$, the probability of the
overestimate is $|C \cap S| > \frac{1+b}{1-a} fk$. We would like to
prove that if (I) and (II) are both false, then so is the probability
of the overestimate. Because $a$ and $b$ have been fixed as pr. the
assignment text, we have that $p=\frac{k}{n(1-a)}$. We also recall
that $n=|A|$.

We assume the opposite of (I) and (II) to be true:
%
\begin{description}
	\item[(I`)] The number of elements from $A$ that hash below $p$
	is greater than $k$.
	\item[(II`)] The number of elements from $C$ that hash below $p$
	is less than or equal to $(1+b) p |C|$.
\end{description}
%
(I`) implies that all elements of $S$ hash below $p$ too, since there are $k$
elements in $S$ and there are more than $k$ elements in $A$ that hash below
$p$. Per (II`), $|C|\leq (1+b)p|C|$ and therefore also 
$|C \cap S| \leq (1+b)p|C|$. \textit{Given} (I`) and (II`), we want to arrive 
at $|C \cap S| \leq \frac{1+b}{1-a} fk$, and if we do so we have proven (I) 
and (II) by contraposition.
%
\begin{align*}
	|C \cap S| 	\leq & (1+b)p|C| & \\
				=& (1+b) \frac{k}{n(1-a)} |C| & \\
				=& (1+b) \frac{k}{|A|(1-a)} |C| & \text{$n=|A|$}\\
				=& (1+b) \frac{k}{1-a} f & \text{Because $f=|C|/|A|$}\\
				=& \frac{(1+b)k}{1-a} f & \\
				=& \frac{1+b}{1-a} fk & 
\end{align*}

\subsection{Exercise 6}
We have that
%
\begin{align}
	k =& \mu_A(1-a) = \mu_A(1 - r \sqrt{k}) \label{eq:6:k} \\
	u_A =& E[X_A] = pn = k/(1-a) \label{eq:6:ua} \\
	& Pr\left[ |X - \mu| \geq r \sqrt{\mu} \right] \leq \frac{1}{r^2} \label{eq:6:pr}
\end{align}
%
We wish to prove that $P_{(I)} = Pr \left[ X_A < k\right] \leq 1/r^2$:

\begin{align*}
	 & Pr\left[X_A < k\right] & \\
	=& Pr\left[X_A < \mu_A(1-\frac{r}{\sqrt{k}})\right] & \text{From \eqref{eq:6:k}} \\
	=& Pr\left[X_A < \mu_A - \frac{\mu_A \cdot r}{\sqrt{k}}\right] & \\
	=& Pr\left[X_A - \mu_A < -\frac{\mu_A \cdot r}{\sqrt{k}} \right] & \\
	=& Pr\left[\mu_A - X_A > \frac{\mu_A \cdot r}{\sqrt{k}} \right] & \text{Multiplying with -1} \\
	\leq & Pr\left[|X_A - \mu_A| > \frac{\mu_A \cdot r}{\sqrt{k}} \right] 
		& \\
	\leq & Pr\left[|X_A - \mu_A| > \frac{\mu_A \cdot r}{\sqrt{\mu_A}} \right]
		& \text{\eqref{eq:6:ua} implies $\mu_A > k$} \\
	=& Pr\left[|X_A - \mu_A| > r \sqrt{\mu_A} \right]
		& \text{Because $\frac{a}{\sqrt{a}} = \sqrt{a}$}\\
	\leq & Pr\left[|X_A - \mu_A| \geq r \sqrt{\mu_A} \right]
		& \\
	\leq & 1/r^2 & \text{See \eqref{eq:6:pr}}
\end{align*}

\subsection{Exercise 7}
We did not finish this assignment in time. We want to end up with Lemma 1 again:

\begin{align*}
	& Pr\left[ X_C > (1+\frac{r}{\sqrt{fk}}) \mu_C \right] & \\
	\leq & Pr\left[ X_C > (1+\frac{r}{\sqrt{\mu_C}}) \mu_C \right]
		& \text{Because $\mu_C > fk$} \\
	=& Pr\left[ X_C > \mu_C + \mu_C \cdot \frac{r}{\sqrt{\mu_C}} \right]
		& \\
	=& Pr\left[ X_C - \mu_C > \mu_C \cdot \frac{r}{\sqrt{\mu_C}} \right]
		&
\end{align*}