\section{Bottom-$k$ sampling}

\subsection{Exercise 2}
We assume that $C$ and $S_h^k(A)$ are independently drawn, and that $S_h^k(A)$ is a uniformly
random subset of $A$ of size $k$. We know that the probability of an element in $A$ being
in the bottom-$k$ sample $S_h^k(A)$ is $p=k/n$.

The probability that a randomly drawn element from $A$ is also in the subset $C$
must be $|C|/|A|$. Intuitively, as $C$ grows in size, so does the probability of picking
an element from $A$ that is also in $C$.

Then we have that
\begin{align}
  & \mathbb{E} \left[ | C \cap S_h^k(A) | \right] \notag \\
  &= \sum_{a\in A} \mathbb{E}\left[ a \in C \land a \in S_h^k(A) \right] \notag \\
  &= \sum_{a\in A} Pr\left[ a \in C \land a \in S_h^k(A) | \right] \notag \\
  &= \sum_{a\in A} \left( Pr\left[ a \in C \right] \cdot Pr\left[ a \in S_h^k(A) \right] \right) 
    \label{eq:2:independent} \\
  &= \sum_{a\in A} \left( \frac{|C|}{|A|} \cdot \frac{k}{|A|} \right) \label{eq:2:final}
\end{align}
where \eqref{eq:2:independent} because the probability of the events is independent, and \eqref{eq:2:final} makes use of \eqref{eq:2:sizeOfC} as well as the probability for an element of
$A$ being in $S_h^k(A)$.

Finally we have that:
\begin{align*}
  & \mathbb{E}\left[ \frac{| C \cap S_h^k(A) |}{k}\right] & \\
  &= \frac{1}{k} \cdot \mathbb{E}\left[ | C \cap S_h^k(A) |\right] & \\
  &= \frac{1}{k} \cdot \sum_{a\in A} \left( \frac{|C|}{|A|} \cdot \frac{k}{|A|} \right) 
    & \quad \text{See \eqref{eq:2:final}} \\
  &= \frac{1}{k} \cdot \frac{|C|}{|A|} \cdot \frac{k}{|A|} \cdot \sum_{a\in A} 1
    & \quad \text{Components of sum do not depend on $a\in A$} \\
  &= \frac{1}{k} \cdot \frac{|C|}{|A|} \cdot \frac{k}{|A|} \cdot |A| & \\
  &= \frac{1}{k} \cdot \frac{|C|}{|A|} \cdot k & \\
%  &= 1 \cdot \frac{|C|}{|A|} \\
  &= \frac{|C|}{|A|}
\end{align*}

\subsection{Exercise 3(a)}
\label{sec:ex3a}
We would store the buttom-$k$ samples in a minimum heap structure $H$, sorted by
their hashing value. This way we can insert new entries in $O(\text{lg } n)$,
and retrieve the $S^{k}_{h}(H)$ lowest hash values in $O(k \text{ lg } n)$ where
$n$ is the total number of input values.

\subsection{Exercise 3(b)}
As written above we would be able to process/insert the next key in $O(\text{lg
} n)$ time.

\subsection{Exercise 4(a)}
We will prove the equality $S^{k}_{h}(\text{A} \cup \text{B}) =
S^{k}_{h}(S^{k}_{h}(A) \cup S^{k}_{h}(B))$.
%
We can see each set as a sorted stack that keeps the smallest values at the
top. The left hand part of the equality ($S^{k}_{h}(\text{A} \cup \text{B}))$)
corresponds to merging the two stacks and taking the $k$ top values.
%
The right hand side ($S^{k}_{h}(S^{k}_{h}(A) \cup S^{k}_{h}(B))$) corresponds to
taking the $k$ topmost values from both stacks and then merging them and taking
the $k$ smallest values from the resulting stack.

Since we take the $k$ smallest values from each stack we are guaranteed to have
the smallest value from the union of $A$ and $B$ since even if the smallest
values in $B$ is bigger than the biggest values in $A$ we still have the $k$
smallest values from $A$, thus proving the equality.
\todo[inline]{Someone should formalize this a bit, XOXO Martin}

\subsection{Exercise 4(b)}
% Left side of the equality is basically taking the k smallest values from the
% union and only keeping those that appear in both sets.

% Right side of the equality is basically taking the k smallest values from the
% union and only keeping those that are part of the k-smallest in both sets.

We want to prove the equality $A \cap B \cap S^{k}_{h}(A \cup B) = S^{k}_{h}(A)
\cap S^{k}_{h}(B) \cap S^{k}_{h}(A \cup B)$.

The lefthandside corresponds to taking the $k$ smallest keys in the union of
$A$ and $B$ and then eliminating all entries that is not also in both $A$ and
$B$.
%
The righthandside is the same as taking the $k$ smallest keys in the union of
$A$ and $B$ and then eliminating all the entries that is not part of the $k$
smallest keys in both $A$ and $B$.

Since $S^{k}_{h}(A \cup B)$ naturally limits both sides of the equality to only
the $k$ smallest entries in the union, $A \cap B$ gets limited by the intersect
with $S^{k}_{h}(A \cup B)$.
\todo[inline]{Someone should formalize this a bit, XOXO Martin}

%We see trivially that $S^{k}_{h}(A) = S^{k}_{h}(A)$. We also easily see that
%$S^{k}_{h}(A) \cap S^{k}_{h}(B) \subseteq A \cap B$

\subsection{Exercise 4(c)}
\todo[inline]{Error check me, plox! XOXO Martin}
We can divide
\[
|S^{k}_{h}(A) \cap S^{k}_{h}(B) \cap S^{k}_{h}(S^{k}_{h}(A) \cup S^{k}_{h}(B))|
/ k
\]
%
into several smaller pieces in order to estimate it's running time, on psudo BNF
it will be
\begin{align*}
  full  &= |sets| / k \\
  sets  &= S^{k}_{h}(A) \cap S^{k}_{h}(B) \cap ext \\
  ext   &= S^{k}_{h}(union) \\
  union &= S^{k}_{h}(A) \cup S^{k}_{h}(B)
\end{align*}
With these smaller fragments it will be easier to calculate the runningtime and
see where the different time consumers are. We will start from the bottom and
move up
\begin{itemize}
\item[$union$] Assuming we're using our minimum-heap structure proposed in
  \ref{sec:ex3a} $S^{k}_{h}(A)$ and $S^{k}_{h}(B)$ can both be done in $O(k
  \text{ lg } n)$. The union can be done in $\Theta(2k)$. Giving this part a
  running time of $O(k \text{ lg } n)) + \Theta(k)$.
%
\item[$ext$] The output from $union$ now contains $2k$ elements meaning we can
  get the $k$ smallest in $O(k \text{ lg } 2k))$ time.
%
\item[$sets$] Each of the extractions in this part can be done in $O(k \text{ lg
  } n)$ time, and since the heaps are ordered the intersect requires only linear
  time giving running time of $O(k \text{ lg } n) + O(k)$. The intersect with
  the $ext$ part requires linear runningtime in the number of elements in the
  largest heap, since both are of maximum size $k$, the final time is $O(k
  \text{ lg } n)) + O(k) + O(k)$.
%
\item[$full$] Since the number of entries in the heap can be decided in $O(1)$
  time if we assume the heap has an associated peorperty with the number of
  nodes that are updated in all operations (also done in constant time), and the
  division itself is also in constant time. This means we just need to determine
  the runningtime of all the above segments combined.
\begin{align*}
  union &= O(k \text{ lg } n)) + \Theta(k) \\
  ext   &= O(k \text{ lg } 2k)) \\
  sets  &= O(k \text{ lg } n)) + O(k) + O(k) \\
  full  &= O(1) + O(k \text{ lg } n)) + O(k) + O(k) + O(k \text{ lg } 2k)) + O(k \text{ lg } n)) + \Theta(k) \\
        &= O(k \text{ lg } n))
\end{align*}
\end{itemize}