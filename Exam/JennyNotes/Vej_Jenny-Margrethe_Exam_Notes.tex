\RequirePackage[l2tabu, orthodox]{nag}
\documentclass[12pt]{article}
\usepackage[utf8]{inputenc} 
\usepackage[T1]{fontenc}
\usepackage[english]{babel} 
\usepackage[margin=2.5cm]{geometry}
\geometry{a4paper}
\usepackage{longtable}


%----------------------------KODE START---------------------------------------
\usepackage{listings}
\usepackage{color}
\usepackage[usenames,dvipsnames]{xcolor}
\definecolor{gray}{rgb}{0.5,0.5,0.5}
\definecolor{mauve}{rgb}{0.58,0,0.82}
\lstset{
  basicstyle=\footnotesize,
  numbers=left,
  numberstyle=\tiny\color{gray},
  stepnumber=1,
  numbersep=10pt,
  backgroundcolor=\color{white},
  showspaces=false,               % show spaces adding particular underscores
  showstringspaces=false,         % underline spaces within strings
  showtabs=false,                 % show tabs within strings adding particular underscores
  frame=single,                   % adds a frame around the code
  rulecolor=\color{black},        
  tabsize=4,
  captionpos=b,                   % sets the caption-position to bottom
  breaklines=true,                % sets automatic line breaking
  breakatwhitespace=false,        % sets if automatic breaks should only happen at whitespace
  title=\lstname,                   % show the filename of files included with \lstinputlisting;
                                  % also try caption instead of title
  keywordstyle=\color{mauve},          % keyword style
  commentstyle=\color{Maroon},       % comment style
  stringstyle=\color{BlueViolet},         % string literal style
  escapeinside={\%*}{*)},            % if you want to add LaTeX within your code
  morekeywords={*,...},              % if you want to add more keywords to the set
  deletekeywords={...}              % if you want to delete keywords from the given language
}
%----------------------------KODE SLUT----------------------------------------


\setlength\parindent{0pt} % Makes \noindent standard
\usepackage{graphicx} 
\usepackage{sidecap}
\usepackage{caption}
\usepackage{subcaption}
\usepackage{float} 
\usepackage{wrapfig} % Allows in-line images if needed
\usepackage{hyperref}
\usepackage{amsmath}
\usepackage{amsfonts}
\usepackage{mathtools}
\hypersetup{colorlinks=false,hidelinks, citecolor=black, urlcolor=black}
\usepackage{csquotes}
\usepackage{comment}

\usepackage[dot, autosize, outputdir="dotgraphs/"]{dot2texi}
\usepackage{tikz}
\usetikzlibrary{shapes}
\usepackage{url}
\usepackage{booktabs}
\usepackage{multirow}
\usepackage{longtable}
\setcounter{secnumdepth}{4}
\setcounter{tocdepth}{4}
\usepackage[titletoc]{appendix} % Names appendices "Appendix A"
                                % instead of just A in Contents
\usepackage[bottom]{footmisc}
\usepackage{pdfpages}
\usepackage{algorithm}% http://ctan.org/pkg/algorithms
\usepackage{algpseudocode}% http://ctan.org/pkg/algorithmicx


%\usepackage{lmodern}
\usetikzlibrary{arrows,automata}
\usepackage{verbatim}

\linespread{1.2} 
\graphicspath{{./figures/}} 

% fancy drawings
\usepackage{pgf}
\usepackage{epigraph}

% \epigraphsize{\small}% Default
\setlength\epigraphwidth{8cm}
\setlength\epigraphrule{0pt}

\usepackage{etoolbox}

\makeatletter
\patchcmd{\epigraph}{\@epitext{#1}}{\itshape\@epitext{#1}}{}{}
\makeatother

%\usepackage{boxproof}
%\usepackage{nomencl}
\usepackage{natbib}

\newcommand{\fasto}{\textsc{Fasto} }
\newcommand{\mips}{\textsc{Mips} }
\newcommand{\mars}{Mars }
\makeatletter
\def\BState{\State\hskip-\ALG@thistlm}
\makeatother

%-----------------------------------------------------------------------------
\begin{document}
\begin{titlepage}

\centering 

\textsc{\Large Advanced Algorithms and Datastructures}\\[0.5cm] 
\textsc{\large Exam Notes}\\[0.5cm] 

\vfill

\emph{Author:}
\\
Jenny-Margrethe \textsc{Vej} -- 250986 -- rwj935
\vspace{20mm}

{\large Block 4, April-June, 2014}\\[3cm] 

\end{titlepage}
%-----------------------------------------------------------------------------

\tableofcontents
\newpage

%-----------------------------------------------------------------------------

%%Completely done 
\input{MaxFlow}
\newpage

%%Missing the last 4 pages in the chapter in the book and the disposition
\section{Fibonacci Heaps}
Okaaaaaay! This is awesome! Kick some serious ass!
%
\subsubsection{Mergeable Heaps}
A mergeable heap is any data structure that supports the following five operations in which each has a key:

\begin{footnotesize}
\begin{itemize}
	\item MAKE-HEAP() creates and returns a new heap containing no elements
	\item INSERT($H, x$) inserts an element $x$, whose $key$ has already been filled in, into heap $H$
	\item MINIMUM($H$) returns a pointer to the element in heap $H$ whose key is minimum
	\item EXTRACT-MIN($H$) deletes the element from heap $H$ whose key is minimum, returning a pointer to the element
	\item UNION($H_1, H_2$) creates and returns a new heap that contains all the elements of heaps $H_1$ and $H_2$. Heaps $H_1$ and $H_2$ are ``destroyed'' by this operation
\end{itemize}
\end{footnotesize}

Besides the mergeable-heap operations above, Fibonacci heaps also support the following two operations:

\begin{footnotesize}
\begin{itemize}
	\item DECREASE-KEY($H, x, k$) assigns to element $x$ within heap $H$ the new key value $k$, which we assume to be no greater than its current value
	\item DELETE($H, x$) deletes element $x$ from heap $H$
\end{itemize}
\end{footnotesize}

\begin{footnotesize}
\begin{table}[H]
\center
\begin{tabular}{l c c}
 & Binary heap & Fibonacci heap \\
 Procedure & (worst case) & (amortised)\\
\hline
MAKE-HEAP 		& $\Phi(1)$ 		& $\Phi(1)$\\
INSERT 			& $\Phi(\lg(n))$ 	& $\Phi(1)$\\
MINIMUM			& $\Phi(1)$		& $\Phi(1)$\\
EXTRACT-MIN		& $\Phi(\lg(n))$	& $O(\lg(n))$\\
UNION			& $\Phi(n)$		& $\Phi(1)$\\
DECREASE-KEY	& $\Phi(\lg(n))$	& $\Phi(1)$\\
DELETE			& $\Phi(\lg(n))$	& $O(\lg(n))$
\end{tabular}
\caption{Running times for operations on two implementations of mergeable heaps. The number of items in the heap(s) at the time of an operation is denoted by $n$}
\end{table}
\end{footnotesize}
%
\subsubsection{Fibonacci heaps in theory and practice}
Fibonacci heaps are theoretically desirable when the number of EXTRACT-MIN and DELETE operations is small relative to the number of other operations performed. This can be the case for some algorithms in graph problems, that call DECREASE-KEY once per edge. Fast algorithms for computing minimum spanning trees and finding single-source shortest paths make essential use of Fibonacci heaps.\\

From a practical point of view, however, the constant factors and complexity of Fibonacci heaps make them less desirable than ordinary binary (or $k$-ary) heaps for most applications. As such, Fibonacci heaps are pre-dominantly of theoretical interest.
%
\subsection{Structure of Fibonacci Heaps}
A Fibonacci Heap is a collection of rooted trees that are min-heap ordered. That is, each tree obeys the min-heap property: the key of a node is greater than or equal to the key of its parent. Each node contains a pointer to its parent and a pointer to any one of its children. The children are linked together in a circular, doubly linked list. In a circular, doubly linked list we can insert a node into any location or remove a node from anywhere in the list in $O(1)$ time. We can also concatenate two such lists in $O(1)$ time.
\begin{figure} [H]
    \begin{center}
        \includegraphics[width=\textwidth]{Images/Hmin.png}
        \caption{Fibonacci heap consisting of five min-heap-ordered trees and 14 nodes}
        \label{fig:H.min}
    \end{center}
\end{figure}

We store the number of children in the cild list of node $x$ in $x.degree$. The boolean-valued attribute $x.mark$ indicates whether node $x$ has lost a child since the last time $x$ was made the child of another node. Newly created nodes are unmarked, and a node becomes unmarked whenever it is made the child of another node. Root nodes are unmarked.\\

We access a Fibonacci heap $H$ by a pointer $H.min$ to the root of a tree containing the minimum key; we call this node the \textit{minimum node} of the Fibonacci heap.\\

The roots of all the trees in a Fibonacci heap are linked together using their left and right pointers into a circular, doubly linked list called the \textit{root list} of the Fibonacci heap. Finally, $H.n$ refers to the number of nodes currently in the heap.
%
\subsubsection{Potential Function}
In the following, the potential function of fibonacci heap is defined by
\begin{align}
	\Phi(H) = t(H) + 2m(H)
\end{align}
where $t(H)$ is the number of trees in the root list and $m(H)$ is the number of marked nodes. The potential function captures the state of a system at a given point, and we can then calculate an amortised\footnote{Amortiseret = Udover n operationer, så er det gennemsnittet} time as the actual time plus a constant times the change in state ‘energy’. This provides a valid upper bound on running time.

\subsubsection{Maximum Degree}
Meeh, nothing important, I think... 

\subsection{Mergeable-heap Operations}
The mergeable-heap operations on Fibonacci heaps delay work as long as possible. We shall see that, no matter what the root list looks like before a EXTRACT-MIN operation, afterward each node in the root list has a degree that is unique within the root list, which leads to a root list of size at most $D(n) + 1$. 
%
\subsubsection{Creating a New Fibonacci Heap}
To make an empty Fibonacci heap, the MAKE-FIB-HEAP procedure allocates and returns the Fibonacci heap object $H$, where $H.n = 0$ and $H.min = NIL$; there are no trees in $H$. Because $t(H) = 0$ and $m(H) = 0$, the potential of the empty Fibonacci heap is $\Phi(H) = 0$. The amortised cost of MAKE-FIB-HEAP is thus equal to its $O(1)$ actual cost.
%
\subsubsection{Inserting a Node}
\begin{figure} [H]
    \begin{center}
        \includegraphics[width=\textwidth]{Images/InsertFib.png}
        \caption{\footnotesize{Inserting a node into a Fibonacci heap. \textbf{(a)} A Fibonacci heap $H$. \textbf{(b)} Fibonacci heap $H$ after inserting the node with key 21. The node becomes its own min-heap-ordered tree and is then added to the root list, becoming the left sibling of the root.}}
    \end{center}
\end{figure}

To determine the amortised cost of FIB-HEAP-INSERT, let $H$ be the input Fibonacci heap and $H'$ be the resulting Fibonacci heap. Then $t(H') = t(H) + 1$ and $m(H') = m(H)$, and the increase in potential is $((t(H) + 1) + 2m(H)) - (t(H) + 2m(H)) = 1$. \\

Since the actual cost is $O(1)$, the amortised cost is $O(1) + 1 = O(1)$.
%
\subsubsection{Finding the Minimum Node}
The minimum node of a Fibonacci heap $H$ is given by the pointer $H.min$, so we can find the minimum node in $O(1)$ actual time. Because the potential of $H$ does not change, the amortised cost of this operation is equal to its $O(1)$ actual cost. 
%
\subsubsection{Uniting Two Fibonacci Heaps}
We simply concatenate the root lists of $H_1$ and $H_2$ and then determines the new minimum node. Afterward, the objects representing $H_1$ and $H_2$ will never be used again. \\

The change in potential is\\ $\Phi(H) - (\Phi(H_1) + \Phi(H_2)) = (t(H) + 2m(H)) - ((t(H_1) + 2m(H_1)) + (t(H_2) + 2m(H_2))) = 0$\\ because $t(H) = t(H_1) + t(H_2)$ and $m(H) = m(H_1) + m(H_2)$. The amortised cost of FIB-HEAP-UNION is therefore equal to its $O(1)$ actual cost. 
%
\subsubsection{Extracting the Minimum Node}
The process of extracting the minimum node is the most complicated of the operations presented in this section. It is also where the delayed work of consolidating trees in the root list finally occurs. \\

FIB-HEAP-EXTRACT-MIN works by first making a root out of each of the minimum node’s children and removing the minimum node from the root list. It then consolidates the root list by linking roots of equal degree until at most one root remains of each degree.\\

Consolidating the root list consists of repeatedly executing the following steps until every root in the root list has a distinct degree value:
\begin{enumerate}
	\item Find two roots $x$ and $y$ in the root list with the same degree. Without loss of generality, let $x.key \leq y.key$
	\item \textbf{\textit{Link}} $y$ to $x$: remove $y$ from the root list, and make $y$ a child of $x$ by calling the FIB-HEAP-LINK procedure. This procedure increments the attribute $x.degree$ and clears the mark on $y$. 
\end{enumerate}

So what we do is, removing $H.min$ and add its children to the root list. Then we start the consolidation with array $A$, starting at the new $H.min$ going to right. When we have to nodes with same degree (height) in a row, we merge them. In the end we have a bunch of nodes at the root list with different degrees, and we mark the new $H.min$. \\

There are several things we must estimate the cost of: extracting the minimum node, the main for-loop through the root list and within that the while-loop that links together trees.\\

The amortised cost of extracting the minimum node of an n-node Fibonacci heap is $O(D(n))$. Let $H$ denote the Fibonacci heap prior to the ExtractMin operation. The actual cost of extracting the minimum node contributes $O(D(n))$ work from processing at most $D(n)$ children of the minimum node.\\

Upon calling consolidate, the size of the root list is at most $D(n) + t(H) - 1$: the original $t(H)$ root-list nodes, minus the extracted root node, plus the children of the extracted node, at most $D(n)$. Every iteration of the inner while-loop, we know that one root is linked to another, and so the total number of iterations of the while loop over all iterations of the for loop is at most the number of roots in the root list. Hence, the total amount of work performed in the for loop is at most proportional to $D(n)+ t(H)$. The total actual work in extracting the minimum node is therefore $O(D(n) + t(H))$. The potential before extracting the minimum node is $t(H) + 2m(H)$. The potential afterwards is at most $(D(n) + 1) + 2m(H)$, since at most $D(n) + 1$ roots remain and no nodes get marked during this operation. The amortised cost is thus at most:
\begin{align*}
	&O(D(n) + t(H)) + ((D(n) + 1) + 2m(H)) - (t(H) + 2m(H))\\
	&= O(D(n)) + O(t(H)) - O(t(H))\\
	&= O(D(n))
\end{align*}
if we scale up the units of the potential to dominate the constant hidden in $t(H)$. We will see later that $D(n) = O(\lg n)$, giving an amortised running time of $O(\lg n)$ for extracting the minimum node.
%
\subsection{Decreasing a Key and Deleting a Node}
\subsubsection{Decreasing a Key}
To decrease a key a few things happens. We start with an initial Fibonacci heap. It is important to notice the marked nodes. A node is now decreased and it is moved to the root list, and its parent is marked. IF the parent was marked already, the marked parent is cut from its own parent and moved to the root list unmarked. When we reach an unmarked node or a root key, the algorithm stops, and we update $H.min$ if necessary. \\

The FIB-HEAP-DECREASE-KEY procedure takes $O(1)$ time plus the time to perform the cascading cuts. Suppose that a given invocation of FIB-HEAP-DECREASE-KEY results in $c$ calls of CASCADING-CUT - each of these calls takes $O(1)$ time exclusive recursive calls. Thus, the actual cost is $O(c)$. \\

Next we compute the change in potential. We let $H$ denote the Fibonacci heap just prior to the FIB-HEAP-DECREASE-KEY operation. The call to CUT creates a new tree rooted at node $x$ and clears $x$'s mark bit. Each call of CASCADING-CUT, except for the last one, cuts a marked node and clears the mark bit. The Fibonacci heap now contains $t(H) + c$ trees and at most $m(H) - c + 2$ marked nodes. The change in potential is therefore at most \\

$((t(H) + c) + 2(m(H) -c + 2)) - (t(H) + 2m(H)) = 4 - c$. \\

Thus, the amortised cost of FIB-HEAP-DECREASE-KEY is at most $O(c) + 4 - c = O(1)$, since we can scale up the units of potential to dominate the constant hidden in $O(c)$. You can now see why we defined the potential function to include a term that is twice the number of marked nodes. When a marked node $y$ is cut by a cascading cut, its mark bit is cleared, which reduces the potential by 2. One unit of potential pays for the cut and the clearing of the mark bit, and the other unit compensates for the unit increase in potential due to node $y$ becoming a root.
%
\subsubsection{Deleting a Node}
Deleting a node works by first setting its key to minus infinity, and then extracting the minimum. We already know that decreasing a key takes $O(1)$ amortised time, and extracting minimum takes $O(D(n))$ amortised time. As we will see that $D(n) = O(\lg n)$, deleting a node runs in amortised time of $O(\lg n)$.

\subsection{Bounding the Maximum Degree}
To prove that the amortised time of FIB-HEAP-EXTRACT-MIN and FIB-HEAP-DELTE is $O(\lg n)$, we must show that the upper bound $D(n)$ on the degree of any node of an $n$-node Fibonacci heap is $O(\lg n)$. In particular, we shall show that $D(n) \leq \lfloor \log_{\phi}n\rfloor$, where $\phi$ is the golden ration, defined in the book as $\phi = \frac{(1 + \sqrt{5})}{2} = 1.61803...$\\

The key to the analysis is as follows. For each node $x$ within a Fibonacci heap, define size($x$) to be the number of nodes, including $x$ itself, in the subtree rooted at $x$. We shall show that size($x$) is exponential in $x.degree$. \\

\textbf{INDSÆT NOGET HER, SOM DU KAN LÆSE OP PÅ, NÅR DU ØVER. DER VIL NÆPPE VÆRE TID TIL AT SNAKKE ET 4 SIDERS OPLÆG IGENNEM MED 5 LEMMAER/COROLLARER}
%
\subsection{Disposition}
\newpage

%%
\section{NP-Completeness}
Who wouldn't love to be examinated in this? GO NUTS! YOU CAN DEFINITELY DO IT!!

\subsubsection{NP-Completeness and the classes P and NP}
Informal description of the three classes - more formal later:\\

\textbf{P:} This class consist of those problems that are solvable in polynomial time. More specifically, they are problems that can be solved in time $O(n^k)$ for some constant $k$, where $n$ is the size of the input to the problem. \\

\textbf{NP:} The class NP consists of those problems that are ``verifiable'' in polynomial time, meaning that if we were somehow given a ``certificate'' of a solution, then we could verify that the certificate is correct in time polynomial in the size of the input to the problem. \\

Any problem in P is also in NP, since if a problem is in P then we can solve it in polynomial time without even being supplied a certificate.\\

\textbf{NPC:} A problem is in the NPC class if it is in NP and as ``hard'' as any problem in NP. If any NP-Complete problem can be solved in polynomial time, then every problem in NP has a polynomial-time algorithm. 

\subsubsection{Overview of showing problems to be NP-Complete}
We rely in three key concepts in showing a problem to be NP-Complete:

\begin{enumerate}
	\item Decision problems vs. optimisation problems 
	\item Reductions
	\item A first NP-Complete problem
\end{enumerate}

\subsection{Polynomial Time}
We begin our study of NP-completeness by formalising our notion of polynomial- time solvable problems. We generally regard these problems as tractable, but for philosophical, not mathematical, reasons. We can offer three supporting arguments.

\begin{enumerate}
	\item Very few practical problems require time on the order of a high-degree polynomial, for example $\Theta(n^{100})$
	\item Second, for many reasonable models of computation, a problem that can be solved in polynomial time in one model can be solved in polynomial time in another 
	\item Third, the class of polynomial-time solvable problems has nice closure proper- ties, since polynomials are closed under addition, multiplication, and composition
\end{enumerate}

\subsubsection{Abstract Problems}
To understand the class of polynomial-time solvable problems, we must first have a formal notion of what a ``problem'' is. We define an abstract problem $Q$ to be a binary relation on a set $I$ of problem instances and a set $S$ of problem solutions. This formulation of an abstract problem is more general than we need for our purposes, since the theory of NP-completeness restricts attention to decision problems: those having a yes/no solution. In this case, we can view an abstract decision problem as a function that maps the instance set $I$ to the solution set \{0, 1\}.

\subsubsection{Encodings}
In order for a computer program to solve an abstract problem, we must represent problem instances in a way that the program understands. An encoding of a set $S$ of abstract objects is a mapping $e$ from $S$ to the set of binary strings.\\

Thus, a computer algorithm that ``solves'' some abstract decision problem actually takes an encoding of a problem instance as input. We call a problem whose instance set is the set of binary strings a concrete problem. We say that an algorithm solves a concrete problem in time $O(T(n))$ if, when it is provided a problem instance $i$ of length $n = |i|$, the algorithm can produce the solution in $O(T(n))$ time. A concrete problem is polynomial-time solvable, therefore, if there exists an algorithm to solve it in time $O(T(n))$ for some constant $k$.\\

We can now formally define the complexity class $P$ as the set of concrete decision problems that are polynomial-time solvable.\\

We would like to extend the definition of polynomial-time solvability from concrete problems to abstract problems by using encodings as the bridge, but we would like the definition to be independent of any particular encoding. That is, the efficiency of solving a problem should not depend on how the problem is encoded. Unfortunately, it depends quite heavily on the encoding if the algorithm runs in either polynomial or superpolynomial time. \\

We cannot really talk about solving an abstract problem without first specifying an encoding. Nevertheless, in practice, if we rule out ``expensive'' encodings such as unary ones, the actual encoding of a problem makes little difference to whether the problem can be solved in polynomial time.\\

\textbf{\textit{Lemma 34.1}} Let $Q$ be an abstract decision problem on an instance set $I$, and let $e_1$ and $e_2$ be polynomially related encodings on $I$. Then, $e_1(Q) \in P$ if and only if $e_2(Q) \in P$.\footnote{I have decided not to focus on all the small proves, but on the proves I think could be relevant for my disposition. The prove for the Lemma 34.1 is on page 1057 in the book}\\

Thus, whether an abstract problem has its instances encoded in binary or base 3 does not affect its ``complexity'', that is, whether it is polynomial-time solvable or not; but if instances are encoded in unary, its complexity may change. We shall assume that the encoding of an integer is polynomially related to its binary representation, and that the encoding of a finite set is polynomially related to its encoding as a list of its elements, enclosed in braces and separated by commas. To denote the standard encoding of an object, we shall enclose the object in angle braces. Thus, $\langle G \rangle$ denotes the standard encoding of a graph $G$.

\subsubsection{A Formal-Language Framework}
By focusing on decision problems, we can take advantage of the machinery of formal-language theory. Let us review some definitions from that theory. An alphabet $\Sigma$ is a finite set of symbols. A language $L$ over $\Sigma$ is any set of strings made up of symbols from $\Sigma$.\\

We denote the empty string by $\epsilon$, the empty language by $\emptyset$, and the language of all strings over $\Sigma$ by $\Sigma^*$. Every language $L$ over $\Sigma$ is a subset of $\Sigma^*$. \\

\textit{The next part in the book is talking about the operations we can do on the languages. I have decided not to make any notes to that. }

\subsection{Polynomial-time Verification}
\subsubsection{Hamiltonian cycles}
The problem of finding a hamiltonian cycle in an undirected graph has been studied for over a hundred years. Formally, a hamiltonian cycle of an undirected graph $G = (V, E)$ is a simple cycle that contains each vertex in $V$. The Hamiltonian-cycle problem is NP-Complete, which we will prove later. 

\subsubsection{Verification Algorithms}
We define a verification algorithm as being a two-argument algorithm $A$, where one argument is an ordinary input string $x$ and the other is a binary string $y$ called a certificate. A two-argument algorithm $A$ verifies an input string $x$ if there exists a certificate $y$ such that $A(x, y) = 1$. The language verified by a verification algorithm $A$ is\\
$L = \{x \in \{0, 1\}^* :\text{ there exists} y \in \{0, 1\}^* \text{ such that} A(x, y) = 1 \}$.

\subsubsection{The complexity Class NP}
The complexity class NP is the class of languages that can be verified by a polynomial-time algorithm. More precisely, a language $L$ belongs to NP if and only if there exist a two-input polynomial-time algorithm $A$ and a constant $c$ such that\\

$ L = \{x \in \{0, 1\}^* : \text{ there exists a certificate} y \text{ with} |y| = O(|x|^c) \text{ such that} A(x, y) = 1$\\

We say that algorithm $A$ verifies language $L$ in polynomial time. 

\subsection{NP-Completeness and Reductability}
Perhaps the most compelling reason why theoretical computer scientists believe that $P \neq NP$ comes from the existence of the class of ``NP-complete'' problems. This class has the intriguing property that if any NP-complete problem can be solved in polynomial time, then every problem in NP has a polynomial-time solution, that is, $P = NP$.

\subsubsection{Reductability}
\textbf{\textit{Lemma 34.3}} \\
If $L_1, L_2 \subseteq \{0, 1\}^* \text{ are languages such that } L_1 \leq_p L_2, \text{ then } L_2 \in P \text{ implies } L_1 \in P $

\subsubsection{NP-completeness}
Polynomial-time reductions provide a formal means for showing that one problem is at least as hard as another, to within a polynomial-time factor. That is, if $L_1 \leq_p L_2$, then $L_1$ is not more than a polynomial factor harder than $L_2$, which is why the ``less than or equal to'' notation for reduction is mnemonic. We can now define the set of NP-complete languages, which are the hardest problems in NP.
A language $L \subseteq \{0, 1\}^*$ is NP-complete if

\begin{enumerate}
	\item $L \in$ NP, and
	\item $L' \leq_p L$ for every $L' \in$ NP
\end{enumerate}

If a language $L$ satisfies property 2, but not necessarily property 1, we say that $L$ is NP-hard. We also define NPC to be the class of NP-complete languages.\\

As the following theorem shows, NP-completeness is at the crux of deciding whether P is in fact equal to NP.\\

\textbf{\textit{Theorem 34.4}} \\
If any NP-complete problem is polynomial-time solvable, then P = NP. Equivalently, if any problem in NP is not polynomial-time solvable, then no NP-complete problem is polynomial-time solvable.

\subsubsection{Circuit Satisfiability}
We now focus on demonstrating the existence of an NP-complete problem: the circuit-satisfiability problem. Boolean combinational circuits are built from boolean combinational elements that are interconnected by wires. A boolean combinational element is any circuit element that has a constant number of boolean inputs and outputs and that performs a well-defined function.\\

The boolean combinational elements that we use in the circuit-satisfiability problem compute simple boolean functions, and they are known as logic gates. The three basic logic gates that we use in the circuit-satisfiability problem is the NOT gate ($\neg$), the AND gate ($\wedge$) and the OR gate ($\vee$). \\

A boolean combinational circuit consists of one or more boolean combinational elements interconnected by wires. A wire can connect the output of one element to the input of another, thereby providing the output value of the first element as an input value of the second.\\

Boolean combinational circuits contain no cycles. In other words, suppose we create a directed graph $G = (V, E)$ with one vertex for each combinational element and with $k$ directed edges for each wire whose fan-out is $k$; the graph contains a directed edge $(u, v)$ if a wire connects the output of element $u$ to an input of element $v$. Then $G$ must be acyclic.\\

The circuit-satisfiability problem is, ``Given a boolean combinational circuit composed of AND, OR, and NOT gates, is it satisfiable?'' In order to pose this question formally, however, we must agree on a standard encoding for circuits.\\

\textbf{\textit{Lemma 34.5}} The circuit-satisfiability problem belongs to the class NP\\
\textbf{\textit{Lemma 34.6}} The circuit-satisfiability problem is NP-hard\\
\textbf{\textit{Theorem 34.7}} The circuit-satisfiability problem is NP-complete \\

 The proves for the last lemmas and theorem is in the book - I do not think I have to prove it for the exam. At least I hope not. 

\subsection{NP-completeness proofs}
In this section, we shall show how to prove that languages are NP-Complete without directly reducing every language in NP to the given language. The following lemma is the basis of our method for showing that a language is NP-Complete. \\

\textbf{\textit{Lemma 34.8}} If $L$ is a language such that $L' \leq_p L$ for some $L' \in NPC$, then $L$ is NP-hard. If, in addition, $L \in$ NP, then $L \in$ NPC.\\

By reducing a known NP-Complete language $L'$ to $L$, we implicitly reduce every language in NP to $L$. Thus Lemma 34.8 gives us a method for proving that a language $L$ is NP-Complete: 

\begin{enumerate}
	\item Prove $L \in$ NP
	\item Select a known NP-Complete language $L'$
	\item Describe an algorithm that computes a function $f$ mapping every instance $x \in \{0, 1\}^*$ of $L'$ to an instance $f(x)$ of $L$
	\item Prove that the function $f$ satisfies $x \in L'$ if and only if $f(x) \in L$ for all $x \in \{0, 1\}^*$
	\item Prove that the algorithm computing $f$ runs in polynomial time
\end{enumerate}

(Steps 2–5 show that $L$ is NP-hard.) This methodology of reducing from a single known NP-Complete language is far simpler than the more complicated process of showing directly how to reduce from every language in NP.

\subsubsection{Formula satisfiability}
Meeeh - skim it again and that is it.

\subsection{NP-Complete Problems}
This section is examples of NP-Complete problems. I have chosen to make notes to only 2 (maybe only one, if I give up before time).

\subsubsection{The Clique Problem}
To long (but read what the problem is)

\subsubsection{The Vertex Cover Problem}
A vertex cover of an undirected graph $G = (V, E)$ is a subset $V' \subseteq V$ such that if $(u, v) \in E$, then $u \in V'$ or $v \in V'$ (or both). That is, each vertex ``covers'' its incident edges, and a vertex cover for $G$ is a set of vertices that covers all the edges in $E$. The size of a vertex cover is the number of vertices in it.\\

\textbf{\textit{Theorem 34.12}}
The Vertex-Cover Problem is NP-Complete\\

\textbf{\textit{Prove:}}

\textit{- First} we prove that VERTEX-COVER $\in$ NP. We have a graph $G = (V, E)$ and an integer $k$. The certificate we choose is the vertex cover $V' \subseteq V$ itself. The verification algorithm affirms that $IV'I = k$ and then it checks, for each edge $(u, v) \in E$, that $u \in V'$ or $v \in V'$. We can easily verify the certificate in polynomial time. \\

\textit{- Next} we prove that the Vertex-Cover Problem is NP-hard by showing that CLIQUE $\leq_p$ VERTEX-COVER. (THIS IS NOT DONE RIGHT NOW - CAN NOT HANDLE CLIQUE 30 MINUTES TO MINDNIGHT ON A FRIDAY)

\subsubsection{The Hamiltonian-Cycle Problem}
To long (but read what the problem is)

\subsubsection{The Traveling-Salesman Problem}
In the traveling-salesman problem, which is closely related to the Hamiltonian-Cycle Problem, a salesman must visit $n$ cities. We model this problem with a complete graph with $n$ vertices, and the salesman wants to make a tour, or Hamiltonian Cycle, visiting each city exactly once, and finishing in the city he starts from. 

\begin{align*}
TSP = \{\langle G, c, k \rangle : 	\text{ } &G = (V, E) \text{ is a complete graph}\\
&c \text{ is a function from } V \text{ x } V \rightarrow \mathbb{Z}\\
&k \in \mathbb{Z}\\
&G \text{ has a traveling-salesman tour with cost at most } k \}
\end{align*}

\textbf{\textit{Theorem 34.14}}
The Traveling-Salesman Problem is NP-Complete\\

\textbf{\textit{Prove:}}\\
\textit{- First} we show that TSP belongs to NP. Given an instance of the problem, we use as a certificate the sequence of $n$ vertices in the tour. The verification algorithm checks that this sequence contains each vertex exactly once, sums up the edge costs, and checks whether the sum is at most $k$. This process can certainly be done in polynomial time.\\

\textit{- Second} we show that TSP is NP-Hard by showing that HAM-CYCLE $\leq_p$ TSP. Let $G = (V, E)$ be an instance of HAM-CYCLE. We construct an instance of TSP as follows. We form the complete graph $G' = (V, E')$, where $E' = \{(i, j) : i, j \in V$ and $i \neq j\}$.\\

The instance of TSP is then $\langle G', c, 0 \rangle$, which we can easily create in polynomial time. \\

\textit{- Last} We show that graph $G$ has a hamiltonian cycle if and only if graph $G'$ has a tour of cost at most 0. Suppose that graph $G$ has a hamiltonian cycle $h$. Each edge in $h$ belongs to $E$ and thus has cost 0 in $G'$. Thus, $h$ is a tour in $G'$ with cost 0. Conversely, suppose that graph $G'$ has a tour $h'$ of cost at most 0. Since the costs of the edges in $E'$ are 0 and 1, the cost of tour $h'$ is exactly 0 and each edge on the tour must have cost 0. Therefore, $h'$ contains only edges in $E$. We conclude that $h'$ is a hamiltonian cycle in graph $G$.


\subsubsection{The Subset-Sum Problem}
To long (but read what the problem is)

\subsection{Disposition}
















\newpage

\section{Randomised Algorithms}
OMG!!! This you can do! Even in danglish

\subsection{Randomised Quicksort}
Vi vil sortere sættet $S$ på $n$ numre. Hvis vi kan finde et $y$ af $S$, sådan at halvdelen er mindre end $y$, så kan vi dele sættet i 2 bortset fra $y$, sådan at $S_1$ indeholder alle elementer, der er $< y$ og $S_2$ resten. Så sorterer vi rekursivt $S_1$ og $S_2$ og outputter elementerne af $S_1$ sorteret stigende, efterfuldt af $y$ og så elementerne af $S_2$, også i stigende rækkefølge. Hvis vi kunne finde $y$ i $cn$ trin for en eller anden konstant $c$m ville vi kunne partionere $S \setminus \{y\}$ til $S_1$ og $S_2$ i $n - 1$ tilsvarende trin ved at sammenligne hvert element af $S$ med $y$ - og dermed ville det totale antal trin være givet af følgende rekurrens: 

\begin{align}
	T(n) \leq 2T(\frac{n}{2}) + (c + 1)n
\end{align}

hvor $T(k)$ repræsenterer tiden det tager for denne metode at sortere $k$ input. Ved direkte substitution kan vi verificere, at $T(n) \leq c'n \lg n$ for en konstant $c'$. \\

Der er en vis udfordring i, at dele $S_1$ og $S_2$ i to lige store dele. Her prøver vi at gøre det med tilfældighed, og ender med algoritmen: RandQS = Randomised Quicksort\footnote{RandQS er et eksempel på en randomiseret algoritme - altså en algoritme, der foretager et tilfældigt valg på et tidspunkt i sin eksekvering - her i step 1}. \\

\textbf{Køretiden for RandQS kan analyseres til følgende:} \\
Køretiden for randomiseret Quicksort er analyseret ud fra antallet af sammenligninger. Vi vil gerne finde the forventede antal af sammenligninger. \\

For $1 \leq i \leq n$, lad $S_{(i)}$ angive elementet af rang $i$ (det $i$te mindste element) i sæt $S$. Således er $S_{(1)}$ det mindste element og $S_{(n)}$ det største. Definér den tilfældige variabel $X_{ij}$ til at antage værdien $1$ hvis $S_{(i)}$ og $S_{(j)}$ er sammenlignet, og værdien $0$ ellers.\\

Det totale antal sammenligninger er 
\[
	\sum_{i=1}^n \sum_{j>i} X_{ij}
\]
Det forventede antal sammenligninger er (by Linearity of Expectation)
\[
	\mathbb{E}[\sum_{i=1}^n \sum_{j>i} X_{ij}] = \sum_{i=1}^n \sum_{j>i} \mathbb{E}[X_{ij}]
\]

Så lader vi $p_{ij}$ angive sansynligheden for at $S_{(i)}$ og $S_{(j)}$ bliver sammenlignet i en udførelse. $X_{ij}$ kan antage værdierne $0$ og $1$, så:
\[
	\mathbb{E}[X_{ij}] = p_{ij} \times 1 + (1 - p_{ij}) \times 0 = p_{ij}
\]

For at bestemme $p_{ij}$, kigger vi på udførelsen af RandQS som et binært søgetræ $T$, hvor hver node er et særskilt element af $S$. Roden i træet er pivoten $y$, som vi valgte indledningsvist. Venstre subtræ indeholder elementerne i $S_1$ og højre subtræ $S_2$. Strukturen bestemmes rekursivt ved sekventielle kørsler af RandQS. Man sammenligner roden af træet $y$ med elementerne i henholdsvis $S_1$ og $S_2$, men elementerne i $S_1$ og $S_2$ sammenlignes ikke med hinanden. \\


This makes $T$ a random binary search tree. We are for this analysis interested in the level-order traversal of the
nodes of $T$. Such a traversal is a permutation $\pi$ obtained by visiting the nodes of $T$ in increasing order of
the level numbers, and in a left-to-right order within each level; recall that the $i$th level of the tree is the set
of all nodes at distance exactly $i$ from the root node.

To compute $p_{ij}$ we make use of two observations:

\begin{enumerate}
	\item There is a comparison between $S_{(i)}$ and $S_{(j)}$ if and only if $S_{(i)}$ or $S_{(j)}$ occur
	earlier in the permutation $\pi$ than any element $S_{(l)}$ such that $i < l < j$. Intuitively, they are
	only compared if there is some pivot node $l$ that puts $i$ in $S_1$ and $j$ in $S_2$. To see this, let
	$S_{(k)}$ be the earliest in $\pi$ from among all elements of rank between $i$ and $j$. If $k \notin \{i,j\}$
	then $S_{(i)}$ will belong to the left sub-tree of $S_{(k)}$ while $S_{(j)}$ will belong to the right sub-tree
	of $S_{(k)}$, implying that there is no comparison between $S_{(i)}$ and $S_{(j)}$. Conversely, if
	$k \in \{i,j\}$ then it must be that there is a parent-child relationship between $S_{(i)}$ and $S_{(j)}$
	implying that the two are compared.

	\item Any of the elements $S_{(i)}, S_{(i+1)}, \hdots, S_{(j)}$ is equally likely to be chosen as the pivot and
	hence appear first in $\pi$. Thus, the probability that the first element is either $S_{(i)}$ or $S_{(j)}$ is
	exactly $2 / (j - i + 1)$.
\end{enumerate}

This establishes that $p_{ij} = 2/(j - i + 1)$. The expected number of comparisons is thus:
%
\begin{align*}
	\sum_{i=1}^n \sum_{j>i} p_{ij} &= \sum_{i=1}^n \sum_{j>i} \frac{2}{j-i+1} \\
	&\leq \sum_{i=1}^n \sum_{k=1}^{n-i+1} \frac{2}{k} \\
	&\leq 2\sum_{i=1}^n \sum_{k=1}^{n} \frac{1}{k}
\end{align*}
%
It follows that the expected number of comparisons is bounded by $2nH_n$, where $H_n$ is the \textit{nth}
Harmonic number, defined by $H_n = \sum_{k=1}^n 1/k$. We know that $H_n \sim ln\ n + \Theta(1)$, making the
expected running time of randomized quicksort $O(n\ lg\ n)$


\subsection{Disposition}
Forslag - bygget på Martin G's noter - måske er Søren D's bedre...

\begin{enumerate}
	\item Randomised Quicksort
	\item Las Vegas, Monte Carlo
	\item 1-sided vs. 2-sided
	\item Bounded Running Times
	\item Markov \& Chebyshev's Inequalities
	\item Randomised Selection
\end{enumerate}
\newpage


%%Haven't looked at those below here

\section{Hashing}
OMG LOLZ'n SHIZZLES! You'r gonna nail this shit!

\subsection{Direct-address tables}
We have a scenario where we want to maintain a dynamic set of some sort. Each element has a key drawn from a universe $U = \{0, 1, ..., m-1\}$ where $m$ is not too large, and no two elements have the same key. \\

Representing it by a direct-address table, or array $T[0...m-1]$, each slot or position corresponds to a key  in $U$. If there is an element $x$ with the key $k$, then $T[k]$ contains a pointer to $x$ - otherwise $T[k]$ is empty represented by $NIL$\\

Illustration in the book on page 254
\begin{algorithm}
\begin{algorithmic}[1]
\State{DIRECT-ADDRESS-SEARCH}{$(T, k)$}
   \State \textbf{return} $T[k]$\\
   
\State{DIRECT-ADDRESS-INSERT}{$T, x$}
   \State \textbf{return} $T[key[x]] \leftarrow x$\\

\State{DIRECT-ADDRESS-DELETE}{$T, x$}
   \State \textbf{return} $T[key[x]] \leftarrow NIL$
\end{algorithmic}
\end{algorithm}

Downsides of direct addressing: If the universe $U$ is large, storing a table $T$ of size $|U|$ may be impractical or even impossible, given the memory available on a typical computer. Furthermore, the set $K$ of keys actually stored may be so small relative to $U$ that most of the space allocated for $T$ would be wasted. 

\subsection{Hash tables}
To overcome the downsides of direct addressing, we can use hash tables. When $K$ is much smaller than $U$, a hash table requires much less space than a direct-address table. It can also reduce storage requirements to $\Phi (|K|)$. \\

This bound is for average-case time whereas for direct addressing it holds for the worst-case time. \\

With direct addressing, an element with $k$ is stored in slot $k$. With hashing, this element is stores in slot $h(k)$; that is, we use a hash function $h$ to compute the slot from the key $k$. There is one hitch: 2 keys may hash to the same slot - called a collision. See page 256 for illustrations. \\

A hash function $h$ must be deterministic in that a given input $k$ should always produce the same output $h(k)$. \\

\textbf{Collision resolution by chaining}\\
In chaining we place all the elements that hash to the same slot into the same linked list. See figure on page 257.

\begin{algorithm}
\begin{algorithmic}[1]
\State{DIRECT-HASH-SEARCH}{$(T, x)$}
   \State \textbf{search} for an element with key $k$ in list $T(T, k)$\\
   
\State{DIRECT-HASH-INSERT}{$T, x$}
   \State \textbf{insert} $x$ at the head of the list $T[h(x.key)]$\\

\State{DIRECT-HASH-DELETE}{$T, x$}
   \State \textbf{delete} $x$ from the list $T[h(x.key)]$
\end{algorithmic}
\end{algorithm}

\textbf{Analysis of hashing with chaining}\\
Given a hash table $T$ with $m$ slots that stores $n$ elements, we define the load factor $\alpha$ for $T$ as $n/m$ which is the same as the average number of elements stores in a chain. \\

Worst case performing: $\Phi (n)$ + the time to compute the hash function (all $n$ keys hash to the same slot, creating a list of length $n$. \\

Average case performing of hashing depends on how well the hash function $h$ distributes the set of keys to be stores among the $m$ slots, on the average. \\

\textbf{\textit{Theorem 11.1}}
In a hash table in which collisions are resolved by chaining, an unsuccessful search takes an average-case time $\Phi (1 + \alpha)$, under the assumption of simple uniform hashing. (OBS: Beviset herfor står på side 259!! LÆS OG FORSTÅ)\\

\textbf{\textit{Theorem 11.2}} 
In a hash table in which collisions are resolved by chaining, a successful search takes average-case time $\Phi (1 + \alpha)$, under the assumption of simple uniform hashing. (OBS: Beviset herfor står på side 259-260!! LÆS OG FORSTÅ)

\subsection{Hash functions}
\textbf{What makes a good hash function?} \\
A good hash function satisfies (approximately) the assumption of simple uniform hashing: each key is equally likely to hash to any of the $m$ slots, independently of where any other key has hashed to. In practice, we can often employ heuristic techniques to create a hash function that performs well. \\

\textbf{Interpreting keys as natural numbers}\\
Most hash functions assume that the universe of keys is the set $\mathbb{N} = \{0, 1, 2, ...\}$ of natural numbers. 

\subsubsection{The division method}
\subsubsection{The multiplication method}
\subsubsection{Universal hashing}
\subsection{Open addressing}
\subsection{Perfect hashing}
\newpage

\section{Exact Exponential Algorithms}
\newpage

\section{Approximation Algorithms}
\newpage

\section{Computational Geometry}
\newpage

\section{Linear Programming and Optimisation}
[\cite{monoids}]

\newpage
\bibliographystyle{plainnat}
%\bibliographystyle{unsrt}
\bibliography{bibliography}


%-----------------------------------------------------------------------------
\end{document}