\subsection{Computational Geometry}
Let $P = \{p_1, p_2, ..., p_n\}$ be a set of points in the plane. To be able to
properly define the triangulation of the plane we first define the ``maximal
planar subdivision'' as a subdivision $\mathcal{S}$ such that no edges that
connects two vertices can be added without destroying the planarity.

A triangulation of $P$ is now defined as a maximal planar subdivision whose
vertex set is $P$.

\begin{description}
\item[Theorem] Let $P$ be a set of $n$ points in the plane, not all collinear,
  and let $k$ denote the number of points in $P$ that lie on the boundary of the
  convex hull of $P$. Then any triangulation of $P$ has $2n−2−k$ triangles and
  $3n−3−k$ edges.

\item[Proof] Let $\mathcal{T}$ be a triangulation of $P$, and let $m$ denote the
  number of triangles of $\mathcal{T}$.  Note that the number of faces of the
  triangulation, which we denote by $n_f$, is $m+1$. Every triangle has three
  edges, and the unbounded face has $k$ edges.  Furthermore, every edge is
  incident to exactly two faces. Hence the total number of edges of
  $\mathcal{T}$ is $n_e = (3m+k)/2$. Euler's formula tells us that
  \[
    n-n_e+n_f = 2
  \]
  Plugging the values for $n_e$ and $n_f$ into the formula, we get $m=2n-2-k$,
  which in turn implies $n_e = 3n-3-k$. \qed
\end{description}


Let $\mathcal{T}$ be a triangulation of $P$, and suppose it has $m$
triangles. Consider the $3m$ angles of $\mathcal{T}$, sorted by increasing
value.  Let $\alpha_1, \alpha_2, ..., \alpha_{3m}$ be the resulting sequence of
angles. We call $A(\mathcal{T}) = (\alpha_1, \alpha_2, ..., \alpha_{3m})$ the
angle vector of $\mathcal{T}$.  Let $\mathcal{T}'$ be another triangulation of
the same point set $P$. We say that $A(\mathcal{T}) > A(\mathcal{T}')$ if
$A(\mathcal{T})$ if there exists and index $i$ with $1 \leq i \leq 3m$ such that
\[
\alpha_j = \alpha_j' \text{ for all } j < i, \text{\quad and \quad} \alpha_i >
\alpha_i'
\]
A triangulation $\mathcal{T}$ is called angle-optimal if $A(\mathcal{T}) \geq
A(\mathcal{T}')$ for all triangulations $\mathcal{T}'$ of $P$.  These are good
because slender triangles make for bad triangulations for terrain.

\textit{Denote the smaller angle defined by three points $p,q,r,$ as
  $\measuredangle pqr$.}

\begin{description}
\item[Thales Theorem] Let $C$be a circle, $\ell$ be a line intersecting $C$ in
  points $a$ and $b$. Let $p, q, r$ and $s$ be points lying on the same side fo
  $\ell$. Suppose that $p$ and $q$ lie on $C$, that $r$ lies inside $C$ nad that
  $s$ lies outside $C$.  Then
  \[
    \measuredangle arb > \measuredangle apb = \measuredangle aqb > \measuredangle asb.
  \]
  \graphicc{0.4}{img/compgeo0.png}{The circle $C$ and the points drwan for
    clarity.}{fig:compgeo0}

\item[Illegal edge] Consider an edge $e = \overline{p_ip_j}$ of a triangulation
  $\mathcal{T}$. If $e$ is not on the unbounded face, it is incident on two
  triangles, $p_ip_jp_k$ and $p_ip_jp_l$. If these triangles form a convex
  quadrilateral, we can obtain a new triangulation $\mathcal{T}'$ by flipping
  the edge. This is done by removing $\overline{p_ip_j}$ and adding
  $\overline{p_kp_l}$. This changes the anglevectors, but only the entries
  associated with the two triangles.  An edgeis considered illegal if flipping
  it increase $A(\mathcal{T})$ such that
  \[
    \underset{1\leq i \leq 6}{\min} \alpha_i < \underset{1\leq i\leq 6}{\min} \alpha_i'.
  \]
  An edge is illegal if we can locally increase the smallest angle simply by flipping it.

\item[Observation 9.3] Let $\mathcal{T}$ be a triangulation with an illegal edge
  $e$. Let $\mathcal{T}'$ be the triangulation obtained from $\mathcal{T}$ by
  flipping $e$. Then $A(\mathcal{T}) > A(\mathcal{T}')$.

\item[Lemma 9.4] Let edge $\overline{p_ip_j}$ be incident on to triangles
  $p_ip_jp_k$ and $p_ip_jp_k$, and let $C$ be the circle through $p_i$, $p_j$
  and $p_k$. The dege $\overline{p_ip_j}$ is illegal iff. the point $p_l$ lies
  in the interior of $C$.  Furthermore, if the points $p_i, p_j, p_k$ and $p_l$
  form a convex quadrilateral and do not lie on a common circle, then exactly on
  of $\overline{p_ip_j}$ and $\overline{p_kp_l}$ is an illegal edge.
  \graphicc{0.4}{img/compgeo1.png}{Visualization of Lemma 9.4.}{fig:compgeo1}

\item[Legal Triangulation] A legal triangulation is simple a triangulation that
  contain no illegal edges. We note that any legal triangulation must also be an
  angle-optimal triangulation. Computing a legal triangulation is simple given
  any triangulation.  simply flip illegal edges until non remain. This is very
  slow though.
\end{description}








\todo[inline]{Help me :( I have no idea what to do x)}