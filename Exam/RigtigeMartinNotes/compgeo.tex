\subsection{Computational Geometry}
Let $P = \{p_1, p_2, ..., p_n\}$ be a set of points in the plane. To be able to
properly define the triangulation of the plane we first define the ``maximal
planar subdivision'' as a subdivision $\mathcal{S}$ such that no edges that
connects two vertices can be added without destroying the planarity.

A triangulation of $P$ is now defined as a maximal planar subdivision whose
vertex set is $P$.

\begin{description}
\item[Theorem] Let $P$ be a set of $n$ points in the plane, not all collinear,
  and let $k$ denote the number of points in $P$ that lie on the boundary of the
  convex hull of $P$. Then any triangulation of $P$ has $2n−2−k$ triangles and
  $3n−3−k$ edges.

\item[Proof] Let $\mathcal{T}$ be a trinagulation of $P$, and let $m$ denote the
  number of triangles of $\mathcal{T}$.  Note that the number of faces of the
  triangulation, which we denote by $n_f$, is $m+1$. Every triangle has three
  edges, and the unbounded face has $k$ edges.  Furthermore, every edge is
  incident to exactly two faces. Hence the total number of edges of
  $\mathcal{T}$ is $n_e = (3m+k)/2$. Euler's formula tells us that
  \[
    n-n_e+n_f = 2
  \]
  Plugging the values for $n_e$ and $n_f$ into the fomula, we get $m=2n-2-k$,
  which in turn implies $n_e = 3n-3-k$. \qed
\end{description}

\todo[inline]{Help me :( I have no idea what to do x)}