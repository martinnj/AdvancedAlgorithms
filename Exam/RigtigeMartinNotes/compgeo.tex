\subsection{Computational Geometry}
Let $P = \{p_1, p_2, ..., p_n\}$ be a set of points in the plane. To be able to
properly define the triangulation of the plane we first define the ``maximal
planar subdivision'' as a subdivision $\mathcal{S}$ such that no edges that
connects two vertices can be added without destroying the planarity.

A triangulation of $P$ is now defined as a maximal planar subdivision whose
vertex set is $P$.

\begin{description}
\item[Theorem] Let $P$ be a set of $n$ points in the plane, not all collinear,
  and let $k$ denote the number of points in $P$ that lie on the boundary of the
  convex hull of $P$. Then any triangulation of $P$ has $2n−2−k$ triangles and
  $3n−3−k$ edges.

\item[Proof] Let $\mathcal{T}$ be a triangulation of $P$, and let $m$ denote the
  number of triangles of $\mathcal{T}$.  Note that the number of faces of the
  triangulation, which we denote by $n_f$, is $m+1$. Every triangle has three
  edges, and the unbounded face has $k$ edges.  Furthermore, every edge is
  incident to exactly two faces. Hence the total number of edges of
  $\mathcal{T}$ is $n_e = (3m+k)/2$. Euler's formula tells us that
  \[
    n-n_e+n_f = 2
  \]
  Plugging the values for $n_e$ and $n_f$ into the formula, we get $m=2n-2-k$,
  which in turn implies $n_e = 3n-3-k$. \qed
\end{description}


Let $\mathcal{T}$ be a triangulation of $P$, and suppose it has $m$
triangles. Consider the $3m$ angles of $\mathcal{T}$, sorted by increasing
value.  Let $\alpha_1, \alpha_2, ..., \alpha_{3m}$ be the resulting sequence of
angles. We call $A(\mathcal{T}) = (\alpha_1, \alpha_2, ..., \alpha_{3m})$ the
angle vector of $\mathcal{T}$.  Let $\mathcal{T}'$ be another triangulation of
the same point set $P$. We say that $A(\mathcal{T}) > A(\mathcal{T}')$ if
$A(\mathcal{T})$ if there exists and index $i$ with $1 \leq i \leq 3m$ such that
\[
\alpha_j = \alpha_j' \text{ for all } j < i, \text{\quad and \quad} \alpha_i >
\alpha_i'
\]
A triangulation $\mathcal{T}$ is called angle-optimal if $A(\mathcal{T}) \geq
A(\mathcal{T}')$ for all triangulations $\mathcal{T}'$ of $P$.  These are good
because slender triangles make for bad triangulations for terrain.

\textit{Denote the smaller angle defined by three points $p,q,r,$ as
  $\measuredangle pqr$.}

\begin{description}
\item[Thales Theorem] Let $C$be a circle, $\ell$ be a line intersecting $C$ in
  points $a$ and $b$. Let $p, q, r$ and $s$ be points lying on the same side fo
  $\ell$. Suppose that $p$ and $q$ lie on $C$, that $r$ lies inside $C$ nad that
  $s$ lies outside $C$.  Then
  \[
    \measuredangle arb > \measuredangle apb = \measuredangle aqb > \measuredangle asb.
  \]
  \graphicc{0.4}{img/compgeo0.png}{The circle $C$ and the points drwan for
    clarity.}{fig:compgeo0}

\item[Illegal edge] Consider an edge $e = \overline{p_ip_j}$ of a triangulation
  $\mathcal{T}$. If $e$ is not on the unbounded face, it is incident on two
  triangles, $p_ip_jp_k$ and $p_ip_jp_l$. If these triangles form a convex
  quadrilateral, we can obtain a new triangulation $\mathcal{T}'$ by flipping
  the edge. This is done by removing $\overline{p_ip_j}$ and adding
  $\overline{p_kp_l}$. This changes the anglevectors, but only the entries
  associated with the two triangles.  An edgeis considered illegal if flipping
  it increase $A(\mathcal{T})$ such that
  \[
    \underset{1\leq i \leq 6}{\min} \alpha_i < \underset{1\leq i\leq 6}{\min} \alpha_i'.
  \]
  An edge is illegal if we can locally increase the smallest angle simply by flipping it.

\item[Observation 9.3] Let $\mathcal{T}$ be a triangulation with an illegal edge
  $e$. Let $\mathcal{T}'$ be the triangulation obtained from $\mathcal{T}$ by
  flipping $e$. Then $A(\mathcal{T}') > A(\mathcal{T})$.

\item[Lemma 9.4] Let edge $\overline{p_ip_j}$ be incident on to triangles
  $p_ip_jp_k$ and $p_ip_jp_k$, and let $C$ be the circle through $p_i$, $p_j$
  and $p_k$. The dege $\overline{p_ip_j}$ is illegal iff. the point $p_l$ lies
  in the interior of $C$.  Furthermore, if the points $p_i, p_j, p_k$ and $p_l$
  form a convex quadrilateral and do not lie on a common circle, then exactly one
  of $\overline{p_ip_j}$ and $\overline{p_kp_l}$ is an illegal edge.
  \graphicc{0.4}{img/compgeo1.png}{Visualization of Lemma 9.4.}{fig:compgeo1}

\item[Legal Triangulation] A legal triangulation is simple a triangulation that
  contain no illegal edges. We note that any legal triangulation must also be an
  angle-optimal triangulation. Computing a legal triangulation is simple given
  any triangulation.  simply flip illegal edges until non remain. This is very
  slow though.
\end{description}



\subsubsection{Delaunay Triangulation}

\begin{codebox}
\Procname{$\proc{DelaunayTriangulation}(P)$}
\li Let $p_0$ be the lexicographically higest point og $P$ (largest $x$ and $y$ coordinate.)
\li Let $p_{-1}$ and $p_{-2}$ be two points in $\mathbb{R}^2$ sufficiently far away such \\
    that $P$ is contained in the triangle $p_0p_{-1}p_{-2}$
\li Initialize $\mathcal{T}$ as the triangulation consisting of $p_0p_{-1}p_{-2}$.
\li Compute a random permutation $p_1,p_2,...,p_n$ of $P$\textbackslash$\{p_0\}$.
\li \For $r \gets 1$ \To $n$ \Do
\li   \Comment{Insert $p_r$ into $\mathcal{T}$}
\li   Find a triangle $p_ip_jp_k \in \mathcal{T}$ containing $p_r$
\li   \If $p_r$ lies on the interior of the triangle \Do
\li     Add edges from $p_r$ to the three vertices $p_ip_jp_k$, splitting it into three triangles.
\li     $\proc{LegalizeEdge}(p_r, \overline{p_ip_j}, \mathcal{T})$
\li     $\proc{LegalizeEdge}(p_r, \overline{p_jp_k}, \mathcal{T})$
\li     $\proc{LegalizeEdge}(p_r, \overline{p_kp_i}, \mathcal{T})$
\li   \Else \Comment{$p_r$ must lie on an edge, say $\overline{p_ip_j}$}
\li     Add edges from $p_r$ to $p_k$ and to the third vertex $p_l$ of the other triangles that is \\
        incident to $\overline{p_ip_j}$, thereby splitting the two triangles incident to \\
        $\overline{p_ip_j}$ into four triangles.
\li     $\proc{LegalizeEdge}(p_r, \overline{p_ip_l}, \mathcal{T})$
\li     $\proc{LegalizeEdge}(p_r, \overline{p_lp_j}, \mathcal{T})$
\li     $\proc{LegalizeEdge}(p_r, \overline{p_jp_k}, \mathcal{T})$
\li     $\proc{LegalizeEdge}(p_r, \overline{p_kp_i}, \mathcal{T})$ \End \End
\li Discard  $p_{-1}$ and $p_{-2}$ and all their incident edges from $\mathcal{T}$.
\li \Return $\mathcal{T}$
\end{codebox}

\begin{codebox}
\Procname{$\proc{LegalizeEdge}(p_r, \overline{p_ip_j}, \mathcal{T})$}
\li \If $\overline{p_ip_j}$ is illegal \Do
\li   Let $p_ip_jp_k$ be the truangle adjacent to $p_rp_ip_j$ along $\overline{p_ip_j}$.
\li   Replace $\overline{p_ip_j}$ with $\overline{p_rp_k}$.
\li   $\proc{LegalizeEdge}(p_r, \overline{p_ip_k}, \mathcal{T})$
\li   $\proc{LegalizeEdge}(p_r, \overline{p_kp_j}, \mathcal{T})$
\end{codebox}

\graphicc{0.8}{img/compgeo2.png}{An example insertion of $p_r$ and a calls to
  legalize edge.}{fig:compgeo2}


\begin{description}
\item[Lemma 9.10] Every new edge created in \texttt{DelaunayTriangulation} or in
  \texttt{LegalizeEdge} during the insertion of $p_r$, is an edge of the
  Delaunay graph of $\{p_{-2},p_{-1},p_0,...,p_r\}$

\item[Proof] Consider first the edges $\overline{p_rp_i}$,
  $\overline{p_rp_j}$,$\overline{p_rp_k}$ (and perhaps $\overline{p_rp_l}$)
  created by splitting $p_ip_jp_k$ (and maybe $p_ip_jp_l$).  Since $p_ip_jp_k$
  is a triangle in the Delaunay triangulation before the addition of $p_r$, the
  circumcirle $C$ of $p_ip_jp_k$ contains no point $p_t$ with $t<r$ in its
  interior.  By shrinking $C$ we can find a circle $C'$ through $p_i$ and $p_r$
  contained in $C$.  Because $C' \subset C$ we know that $C'$ is empty. This
  implies that $\overline{p_rp_i}$ is an edge of the Delaunay graph after the
  addition of $p_r$. The same holds for $\overline{p_rp_j}$ and
  $\overline{p_rp_k}$ (and $\overline{p_rp_l}$ if it exists).

Now consider and edge flipped by \texttt{LegalizeEdge}. Such an edge flip always
replaces an edge $\overline{p_ip_j}$ of a triangle $p_ip_jp_l$ by and edge
$\overline{p_rp_l}$ incident to $p_r$.  Since $p_ip_jp_l$ was a Delaunay
triangle before the addition of $p_r$ and becauseits circumcircle $C$ contains
$p_r$ -- otherwise $\overline{p_ip_j}$ would not be illegal -- we can shrink the
circumcircle to obtain an empty circle $C'$ with only $p_r$ and $p_l$ on its
boundary. Hence $\overline{p_rp_l}$ is an edge of the Delaunay graph after the
addition \qed

\graphicc{0.4}{img/compgeo3.png}{Helpfull illustration for the proof. :)}{fig:compgeo3}

\end{description}

\todo[inline]{Look op the stuff on p. 202 of the comp geo paper, it's prett smart with pointers and trees.}