\subsection{Randomized Algorithms}

When talking random variables there are generally 3 kinds:
\begin{enumerate*}
  \item Las Vegas algorithms
  \item Monte Carlo algorithms with one-sided error
  \item Monte Carlo algorithms with two-sided error
\end{enumerate*}

A Las Vegas algorithm is a randomized algorithm that have zero possibility of
producing an invalid solution but where the running time is affected vy the
randomization.

A Monte Carlo algorithm is a randomized algorithm that might produce an
incorrect solution. For decisions problems these can be one-sided or
two-sided. A one sided algorithm is always correct for one of the answer
(yes/no) but might be wrong on the other one. If it is two-sided then it might
be wrong on both answers.

% \subsubsection{Randomized Quicksort}
% \begin{description}
% \item[Input:] A set of numbers, $S$.
% \item[Output:] The elements of $S$ sorted in increasing order.
% \item[1.] Choose an element $y$ uniformly at random from $S$: Every element in
%   $S$ have equal probability of being chosen.
% \item[2.] By comparing each element of $S$ with y, determine the set $S_1$ of
%   elements smaller than $y$ and the set $S_2$ of elements larger than $y$.
% \item[3.] Recursively sort $S_1$ and $S_2$. Output the sorted version of $S_1$
%   followed by $y$, then the sorted version of $S_2$.
% \end{description}

% To find the running time of the algorithm we need to find the expected number of
% comparisons done in step 2. For $1 \leq i \leq n$, let $S_{(i)}$ denote the
% element of rank $i$ (the $i$th smallest element) in the set $S$.  Thus $S_{(1)}$
% denotes the smallest element of $S$, and $S_{(n)}$ denotes the largest.  Define
% a random variable $X_{ij}$ to assume the value 1 if $S_{(i)}$ and $S_{(j)}$ are
% compared in an execution, and the value 0 otherwise.  Thus, $X_{(ij)}$ is a
% count of comparisons between $S_{(i)}$ and $S_{(j)}$, and so the total number of
% comparisons is $\sum^n_{i=1}\sum_{j>i} X_{ij}$. We want the expected number of
% comparisons which is:
% \begin{align}
% \mathbb{E}[\sum^n_{i=1}\sum_{j>i} X_{ij}] = \sum^n_{i=1}\sum_{j>i}
% \mathbb{E}[X_{ij}] \label{eq:rand1.2}
% \end{align}
% Let $p_{ij}$ denote the probability that $S_{(i)}$ and $S_{(j)}$ are compared in
% an execution.  Since $X_{ij}$ only assumes the values 0 and 1,
% \begin{align}
%   \mathbb{E}[X_{ij}] = p_{ij} \times 1 + (1 - p_{ij}) \times 0 = p_{ij} \label{eq:rand1.3}
% \end{align}

% To faciltiate the determination of $p_{ij}$, we view the execution of the
% algorithm as a binary tree $T$, each node is labeled as an element of $S$, with
% $y$ as the root, and the left subtree being $S_1$ and the right subtree being
% $S_2$.  The root $y$ is compared to the elements in the two sub-trees, but no
% comparison is performed between elements in different subtrees.  Thus there is a
% comparison between $S_{(i)}$ and $S_{(j)}$ iff one of the elements is an
% ancestor of the other.

% The in-order traversal of $T$ will visit the elements of $S$ in sorted order,
% and that is the output of the algorithm.

% The level-order traversal (left to right then one level down and repeat) yields
% a permutation $\pi$.  To compute $p_{ij}$ we make two observations:
% \begin{enumerate*}
% \item There is a comparison between $S_{(i)}$ and $S_{(j)}$ iff $S_{(i)}$ or
%   $S_{(j)}$ occurs earlier in the permutation $\pi$ than any element
%   $S_{(\ell)}$ such that $i < \ell < j$. To seee this let $S_{(k)}$ be the
%   earliest in $\pi$ from among all elements of rank between $i$ and $j$.  If $k
%   \notin \{ i,j \}$, then $S_{(i)}$ wil lbelong to the left sub-tree of
%   $S_{(k)}$ while $S_{(j)}$ will belong to the right. implying that there is no
%   comparison between the two.  Conversely if $k \in \{i,j\}$ , there is an
%   ancesto-descendant relationship between $S_{(i)}$ and $S_{(j)}$, implying
%   comparisons.

% \item Any of the elements $S_{(i)}$, $S_{(ji+1}$, ..., $S_{(j)}$ is equally
%   likely to be the first element chosen as partition element, and hence appear
%   first in $\pi$.  Thus the probability that this first element is either
%   $S_{(i)}$ or $S_{(j)}$ is exactly $\frac{2}{j-i+1}$.
% \end{enumerate*}

% We have established that $p_{ij} = \frac{2}{j-i+1}$ and by equations
% \ref{eq:rand1.2} and \ref{eq:rand1.3} the expected number of comparisons is
% given by:
% \begin{align*}
% \sum^n_{i=1} \sum_{j>i} p_{ij} &= \sum^n_{i=1} \sum_{j>i} \frac{2}{j-i+1} \\
%  &\leq \sum^n_{i=1} \sum^{n-i+1}_{k=1} \frac{2}{k} \\
%  &\leq 2\sum^n_{i=1} \sum^{n}_{k=1} \frac{1}{k}
% \end{align*}

% It follow that the expected number of comparisons is bounded above by $2nH_n$,
% where $H_n$ is the n'th harmonic number, defined by: $H_n = \sum^n_{k=1} 1/k$.

% \begin{description}
% \item[Theorem 1.1] The expected number of comparisons in the execution of
%   randimized quicksort is at most $2nH_n$.

% \item[Proof] \qed
% \end{description}

% \subsubsection{Randomized selection with LazySelect}
% We can use random sampling to selecth the $k$th smallest element i na set $S$ of
% $n$ elements. We assume all elements in $S$ are distinct. Let $r_s(t)$ denote
% the rank of an element $t$ (the $k$th smallest element has rank $k$) and let
% $S_{(i)}$ denote the $i$th smallest element of $S$. Thus we seek to identify
% $S_{(k)}$.

% \begin{description}
% \item[Input:] A set $S$ of $n$ elements from a totally ordered universe, and an
  % integer $0 < k\leq n$.
% \item[Output:] The $k$'th smallest element of $S$, $S_{(k)}$.
% \item[1.] Pick $n^{3/4}$ elements form $S$, chosen independtly and uniformly at
  % random; calls this multiset of elements $R$.
% \item[2.] Sort $R$ in $O(n^{3/4} log n)$ steps using an optimal sorting
  % algorithm.
% \item[3.] Let $x = kn^{-1/4}$. For $\ell = \text{max}\{\lfloor
  % x-\sqrt{n}\rfloor,1\}$, and $h = \text{min}\{\lceil x + \sqrt{n}\rceil
  % ,n^{3/4}\}$, let $a = R_{(\ell)}$, and $b = R_{(h)}$. By comparing $a$ and $b$
  % to every element o $S$, determine $r_S(a)$ and $r_S(b)$.
% \item[4.] \begin{codebox}
% \li \If $k < n^{1/4}$ let $P \gets \{y \in S | y \leq b\}$ \\
% \li \Else \If $k > n-n^{1/4}$, let $P \gets \{y \in S | y\geq a\}$ \\
% \li \Else \If $k \in [n^{1/4}, n-n^{1/4}]$, let $P \gets \{y \in S | a \leq y \leq b\}$ \\
% \end{codebox}
% Check wether $S_{(k)} \in P$ and $|P| \leq 4n^{3/4}+2$. If not, repeat steps 1-3
% until such a $P$ is found.
% \item[5.] By sorting $P$ in $O(|P|\text{log}|P|)$ steps, identify
  % $P_{(k-r_s(a)+1)}$, which is $S_{(k)}$.
% \end{description}

\subsubsection{Randomized Min Cut}

Events $\varepsilon_1$ and $\varepsilon_2$ are independent if the probability that they both occur are
\begin{align}
  \mathbf{Pr}[\varepsilon_1 \cap  \varepsilon_2] = \mathbf{Pr}[\varepsilon_1] \times \mathbf{Pr}[\varepsilon_2]
\end{align}

When they are not necessarily independent,
\begin{align}
  \mathbf{Pr}[\varepsilon_1 \cap \varepsilon_2] = \mathbf{Pr}[\varepsilon_1|\varepsilon_2] \times
  \mathbf{Pr}[\varepsilon_2] = \mathbf{Pr}[\varepsilon_2|\varepsilon_1] \times \mathbf{Pr}[\varepsilon_1],
  \label{eq:rand2}
\end{align}
sometimes when a collection of events is not independent, a generalization of (\ref{eq:rand2}) can be used:
\begin{align}
  \mathbf{Pr}[\cap^k_{i=1}\varepsilon_i] = \mathbf{Pr}[\varepsilon_1] \times \mathbf{Pr}[\varepsilon_2|\varepsilon_1]
  \times \mathbf{Pr}[\varepsilon_3|\varepsilon_1 \cap \varepsilon_2] ... \mathbf{Pr}[\varepsilon_k|\cap^{k-1}_{i=1}\varepsilon_i].
\end{align}


Consider a multigraph $G$ (multigraphs can have multiple edges between two vertices). A cut in the graph $G$is the removal of a set of edges that divides the graph into
two or more components. A min cut is a cut of minimum cardinality (cardinality is how many items are in a set).
The algorithm works by contracting edges, contracting and edges simply means merge the two vertices at the end and retain all edges connected to them, delete any self-loops.
This algorithm will work since all edges in the newly created graph will always be edges also in $G$ so any cut in $G'$ is also a cut in $G$.

The algorithm is simple:
\begin{enumerate*}
  \item Pick an edge uniformly at random and contract it to create $G'$ with at least one edge less than $G$.
  \item Repeat step 1 until only two vertices remain.
  \item The edges between the two vertices in $G'$ is output as a candidate min-cut.
\end{enumerate*}




\subsubsection{Markov \& Chebyshev's inequalities}
\begin{description}
\item[Markov inequality] Let $Y$ be a random variable assuming only non-negative
  values. Then for all $t \in \mathbb{R}^+$,
  \[
    \text{Pr}[Y \geq t] \leq \frac{\text{E[Y]}}{t}.
  \]
  Equivalently,
  \[
    \text{Pr}[Y \geq k\text{E}[Y]] \leq \frac{1}{k}.
  \]
\item[Proof] Define a function $f(y)$:
  \[
   f(y) = \begin{cases}
     1 & \text{iff } y \geq t\\
     0 & \text{otherwise.}
   \end{cases}
  \]
  Then $\text{Pr}[Y \geq t] = \text{E}[f(y)]$. Since $f(y) \leq y/t$ for all $y$,
  \[
    \text{E}[f(Y)] \leq \text{E}\left [\frac{Y}{t} \right ] = \frac{\text{E}[Y]}{t},
  \]
  and the theorem follows. \qed
\end{description}

\begin{description}
\item[Chebyshevs inequality] Let $X$ be a random variable with expectation
  $\mu_X$ and a standard deviation of $\sigma_X$. Then for any $t \in
  \mathbb{R}^+$,
  \[
    \text{Pr}[|X - \mu_X| \geq t\sigma_X] \leq \frac{1}{t^2}
  \]
\item[Proof] Note that
  \[
    \text{Pr}[|X - \mu_X| \geq t\sigma_X] = \text{Pr}[(X - \mu_X)^2 \geq
    t^2\sigma_X^2]
  \]
  The random variable $Y = (X - \mu_X)^2$ has expectation $\sigma_X^2$, and
  applying the Markov inequality to $Y$ bounds this probability from above by
  $1/t^2$. \qed
\end{description}
