\subsection{Randomized Algorithms}

When talking random variables there are generally 3 kinds:
\begin{enumerate*}
  \item Las Vegas algorithms
  \item Monte Carlo algorithms with one-sided error
  \item Monte Carlo algorithms with two-sided error
\end{enumerate*}

A Las Vegas algorithm is a randomized algorithm that have zero possibility of
producing an invalid solution but where the running time is affected vy the
randomization.

A Monte Carlo algorithm is a randomized algorithm that might produce an
incorrect solution. For decisions problems these can be one-sided or
two-sided. A one sided algorithm is always correct for one of the answer
(yes/no) but might be wrong on the other one. If it is two-sided then it might
be wrong on both answers.


\subsubsection{Markov \& Chebyshev's inequalities}
\begin{description}
\item[Markov inequality] Let $Y$ be a random variable assuming only non-negative
  values. Then for all $t \in \mathbb{R}^+$,
  \[
    \text{Pr}[Y \geq t] \leq \frac{\text{E[Y]}}{t}.
  \]
  Equivalently,
  \[
    \text{Pr}[Y \geq k\text{E}[Y]] \leq \frac{1}{k}.
  \]
\item[Proof] Define a function $f(y)$:
  \[
   f(y) = \begin{cases}
     1 & \text{iff } y \geq t\\
     0 & \text{otherwise.}
   \end{cases}
  \]
  Then $\text{Pr}[Y \geq t] = \text{E}[f(y)]$. Since $f(y) \leq y/t$ for all $y$,
  \[
    \text{E}[f(Y)] \leq \text{E}\left [\frac{Y}{t} \right ] = \frac{\text{E}[Y]}{t},
  \]
  and the theorem follows. \qed
\end{description}

\begin{description}
\item[Chebyshevs inequality] Let $X$ be a random variable with expectation
  $\mu_X$ and a standard deviation of $\sigma_X$. Then for any $t \in
  \mathbb{R}^+$,
  \[
    \text{Pr}[|X - \mu_X| \geq t\sigma_X] \leq \frac{1}{t^2}
  \]
\item[Proof] Note that
  \[
    \text{Pr}[|X - \mu_X| \geq t\sigma_X] = \text{Pr}[(X - \mu_X)^2 \geq
    t^2\sigma_X^2]
  \]
  The random variable $Y = (X - \mu_X)^2$ has expectation $\sigma_X^2$, and
  applying the Markov inequality to $Y$ bounds this probability from above by
  $1/t^2$. \qed
\end{description}

\todo[inline]{Write about lazy select or randomized quicksort}