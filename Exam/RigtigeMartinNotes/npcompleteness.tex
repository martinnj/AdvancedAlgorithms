\subsection{NP-Completeness}
Example of a problem in the $P$ class is an Euler Tour(a path in a graph that
uses all edges exactly once, vertices can be visited multible times) of a graph,
it can be done in $O(E)$. An $NP$ class example that is very similar is a
Hamiltonian cycle. A Hamiltonian cycle is a path that visits all vertices once.

We have three classes of problems in this subject:
\begin{itemize*}
\item[\textbf{P}] Problems that are solvable in polynomial time ($O(n^k)$ for
  some constant $k$.)
\item[\textbf{NP}] Problems that are verifiable on polynomial time, i.e. if we
  have a certificate/solution, can we check it in polynomial time. All problems
  in P will also be in NP.
\item[\textbf{NPC}] A subclass of NPC problems that are ``at least as hard as
  the hardest problems in NP'', these cannot be verified in polynomial time.
\end{itemize*}

\subsubsection{Decision problems vs. optimization problems}
NP-completeness does not cover optimization problems, only decision problems. We
can however use the relationship between optimization and decision problems to
guage if a optimization problems is in fact NP-complete.

The shortest-path problem is an opimization problem, but can converted (in
polynomial time) to a decision problem if the question is posed like so: ``Does
a path $p$ in the graph $G$ exist with only $k$ edges?'', then we iterate over
$k$ and will be able to guage the shortest bath problem as a dicision problem.

\todo[inline]{Continue from p. 1051}
