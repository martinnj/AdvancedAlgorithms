\subsection{NP-Completeness}
Example of a problem in the $P$ class is an Euler Tour(a path in a graph that
uses all edges exactly once, vertices can be visited multible times) of a graph,
it can be done in $O(E)$. An $NP$ class example that is very similar is a
Hamiltonian cycle. A Hamiltonian cycle is a path that visits all vertices once.

We have three classes of problems in this subject:
\begin{itemize*}
\item[\textbf{P}] Problems that are solvable in polynomial time ($O(n^k)$ for
  some constant $k$.)
\item[\textbf{NP}] Problems that are verifiable on polynomial time, i.e. if we
  have a certificate/solution, can we check it in polynomial time. All problems
  in P will also be in NP.
\item[\textbf{NP-Hard}] A subclass of NPC problems that are ``at least as hard
  as the hardest problems in NP'', these cannot be verified in polynomial time.
\item[\textbf{NPC}] Problems that are both in the set of NP problems and NP-hard
  problems. (NP $\cap$ NP-Hard)
\end{itemize*}

\subsubsection{Decision problems vs. optimization problems}
NP-completeness does not cover optimization problems, only decision problems. We
can however use the relationship between optimization and decision problems to
guage if a optimization problems is in fact NP-complete.

The shortest-path problem is an opimization problem, but can converted (in
polynomial time) to a decision problem if the question is posed like so: ``Does
a path $p$ in the graph $G$ exist with only $k$ edges?'', then we iterate over
$k$ and will be able to guage the shortest bath problem as a dicision problem.

\subsubsection{Reduction}
The notion of showing that one problem us no harder or no easier than another
problem applies even when both problems are decision problems. This is used in
almost all NP-Completeness proof as follows: Take an instance $\alpha$ of
problem $A$, that is, a specific input for the problem $A$, so for shortest path
we may choose a graph $G$, and vertices $u$ and $v$ as well as a $k$. Make a
polynomial time transofmration from $\alpha$ to instance $\beta$ of problem $B$
(which can be decided in polynomial time) with the following characteristics:
\begin{enumerate*}
\item Transformation takes polynomial time.
\item The answers are the same. The answer for $\alpha$ is ``yes'', iff. the
  answer for $\beta$ is ``yes'', same for ``no''.
\end{enumerate*}
Such an algorithm is called a reduction algorithm.

We can now solve any instance of $A$ in polynomial time by converting $\alpha$
to $\beta$ in polynomial time, running the polynomial time decision algorithm
for $B$ and using the answer for $B$ as the answer for $A$.

\todo[inline]{Continue from p. 1052-3 something.}
