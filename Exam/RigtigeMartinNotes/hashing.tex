\subsection{Hashing}

\subsubsection{Universal Hashing}
We wish to generate a function $h : U \rightarrow [m]$ from a key universe $U$
to a set of hash values $[m] = \{ 0, ..., m-1 \}$.  We want $h$ to be universal,
so that for any given distinct keys $x,y \in U$, when $h$ is picked at random
(independant from $x$ and $y$), we have a low collision probability:
\[
  \underset{h}{\Pr}[h(x) = h(y)] \leq 1/m.
\]
For many applications it suffices for some constant $c = O(1)$, we have
\[
  \underset{h}{\Pr}[h(x) = h(y)] \leq c/m.
\]
Then $h$ is called $c$-universal.

\subsubsection{Simple Hash Tables w. Chaining}
We have a set $S \subseteq U$ of keys that we wish to store and be able to
retrieve in expected constant time.  Let $n = |S|$ and $m \geq n$. We then pick
a universal hash function $h : U \rightarrow [m]$, and create an array $L$ og
$m$ lists/chains so that for $i\in[m]$, $L[i]$ is the list of keys that has to
$i$.  To see if a key $x\in U$ is in $S$, we check if $x$ is in the list
$L[h(x)]$. This takes time propotional to $1+|L[h(x)]|$.  We add one for the
constant time to look up the list, and then add the number of elements in the
list itself since we need to walk through them.

If $x \notin S$ and $h$ is universal, then the expected number of elements in
$L[h(x)]$ is
\[
  E[|L[h(x)]|] = \sum_{y \in S} \Pr[h(y) = h(x)] = n/m \leq 1
\]

\subsubsection{Signature Hashes}
Another application is to assign a unique signature $s(x)$ to each key. For this
we want $s(x) \neq s(y)$ for all distinct keys $x,y \in S$.  We pick a universal
hash function $s: U \rightarrow [n^3]$. The probability of collision is
calculated assign
\[
  \underset{s}{\Pr}[\exists\{x,y\}\subseteq S : s(x) = s(y)] \leq \sum_{y\in S}
  \underset{s}{\Pr}[s(x) = s(y)] = \binom{n}{2}/n^3 < 1/(2n)
\]
The first inequality is a ``union bound'', the probability of that at least one
of multiple events happen is at most the sum of their probabilities.

\subsubsection{Multiply-mod-prime}
Multiply-mod-prime is an implementation of a hashing function, note that if $m
\geq u$ we can let $h$ be the identity function, since we won't need randomness
to avoid collision, so we assyme $m < u$.

Chose a prime number $p \geq u$. Uniformly random pick $a \in [p]_+ = \{1, ...,
p-1\}$ and $b \in [p] = \{ 0, ..., p-1\}$, and define $h_{a,b}(x) : [u]
\rightarrow [m]$ as
\begin{align}
  h_{a,b}(x) = ((ax+b) mod p) mod m) \label{eq:h1}
\end{align}
Given distinct $x,y \in [u] \subseteq [p]$, we argue that for random $a$ and
$b$, the following is true,
\begin{align}
  \underset{a\in [p]_+,b\in [p]}{\Pr}[h_{a,b}(x) = h_{a,b}(y)] \leq
  1/m. \label{eq:h2}
\end{align}
For most of the proof we consider all $a \in [p]$, including $a=0$. Ruling out
$a=0$, will be used in the end to get a tight bound from (\ref{eq:h2}).

\begin{description}
\item[Fact 2.1] If $p$ is prime and $\alpha, \beta \in [p]_+$ then $\alpha \beta
  \not\equiv 0$ (mod $p$).
\end{description}

For a hiven pair $(a,b) \in [p]^2$, define $(q,r) \in [p]^2$ by
\begin{align}
  ax+b\mod p &= q \label{eq:h3} \\
  ay+b\mod p &= r \label{eq:h4}
\end{align}

\begin{description}
\item[Lemma 2.2] Equations (\ref{eq:h3}) and (\ref{eq:h4}) define a 1-1
  correspondence between pairs $(a,b)\in [p]^2$ and pairs $(r,q)\in[p]^2$.
\item[Proof] For a given pair $(q,r) \in [p]^2$, we will show that there is at
  most one pair $(a,b) \in [p]^2$, satisfying (\ref{eq:h3}) and (\ref{eq:h4}).
  Subtracting (\ref{eq:h3}) from (\ref{eq:h4}) modulo $p$, we get
  \begin{align}
    (ay+b) - (ax+b) \equiv a(y-x) \equiv q-r \quad (\mod p), \label{eq:h5}
  \end{align}
  We claim that there is at most one $a$ satisfying (\ref{eq:h5}). Suppose that
  there is another $a'$ satisfying (\ref{eq:h5}).  Subtracting the equations
  with $a$ and $a'$, we get
  \begin{align*}
    (a-a')(y-x) \equiv 0 (\mod p),
  \end{align*}
  but since $a-a'$ and $y-x$ are both non-zero modulo $p$, this contradicts Fact
  2.1. There is thus at most one $a$ satisfying (\ref{eq:h5}) for given
  $(q,r)$. With this $a$, we need $b$ to satisfy (\ref{eq:h3}), and this
  determines $b$ as
  \begin{align}
    b = ax-q\mod p. \label{eq:h6}
  \end{align}
  Thus, satisfying (\ref{eq:h3}) and (\ref{eq:h4}), each pair $(q,r)\in [p]^2$
  thus corresponds to at most one pair $(a,b)\in [p]^2$. On the other hand,
  (\ref{eq:h3}) and (\ref{eq:h4}) define a unique pair $(r,q)\in[p]^2$ for each
  pair $(a,b)\in [p]^2$. We have $p^2$ pairs of each kind so the correspondence
  must be 1-1.\qed
\end{description}

Since $x \not\equiv y$, by Fact 2.1
\begin{align}
  r = q \Longleftrightarrow a = 0. \label{eq:h7}
\end{align}
Thus we pick $(a,b) \in [p]_+ \times [p]$, we get $r \not\equiv q$.

Returning to the proof of (\ref{eq:h2}), we get a collision if and only if $q
\equiv r (\mod m)$, and we know that $q \neq r$. For a given $r$ there is at
most $\lceil p/m\rceil$ values of $q$ with $q \equiv r (\mod m)$; namely $r$,
$r+m$, 4r+2m 4, ... Ruling out $q = r$ leaves us at most $\lceil p/m \rceil - 1$
valus of $q$ for each of the $p$ values of $r$. Noting that
\[
  \lceil p/m\rceil - 1 \leq  \frac{(p+m-1)}{m}-1 = \frac{p-1}{m},
\]
we get that the total number of collision pairs $(q,r)$, $r \neq q$, is bounded
by at most $p(p-1)/m$. Since each of the $(pp-1)$ pairs from $[p]_+ \times [p]$
are equally likely, we conclide that the collision probability is bounded by
$1/m$, as required for universality.


\subsection{Strong Universality}
For $h : [u] \rightarrow [m]$ we consider pair-wise events of the form for given
distinct keys $x,y \in [u]$ and possibly non-distinct hash values $q,r \in [m]$,
we have $h(x) = q$ and $h(y) = r$. We say a random hash function $h : [u]
\rightarrow [m]$ is storngly universal if the probability of every pair-wise
event is $1/m^2$. If $h$ is Strongly Universal it is also universal since:
\[
  \Pr[h(x) = h(y)] = \sum_{q\in [m]} \Pr[h(x) = q \land h(y) = q] = m/m^2  = 1/m.
\]

\todo[inline]{Continue from page 6 of the hashing paper.}
