\subsection{Hashing}

\subsubsection{Universal Hashing}
We wish to generate a function $h : U \rightarrow [m]$ from a key universe $U$
to a set of hash values $[m] = \{ 0, ..., m-1 \}$.  We want $h$ to be universal,
so that for any given distinct keys $x,y \in U$, when $h$ is picked at random
(independant from $x$ and $y$), we have a low collision probability:
\[
  \underset{h}{\Pr}[h(x) = h(y)] \leq 1/m.
\]
For many applications it suffices for some constant $c = O(1)$, we have
\[
  \underset{h}{\Pr}[h(x) = h(y)] \leq c/m.
\]
Then $h$ is called $c$-universal.

\subsubsection{Simple Hash Tables w. Chaining}
We have a set $S \subseteq U$ of keys that we wish to store and be able to retrieve in expected constant time.
Let $n = |S|$ and $m \geq n$. We then pick a universal hash function $h : U \rightarrow [m]$, and create an array $L$ og $m$ lists/chains so that for $i\in[m]$,
$L[i]$ is the list of keys that has  to $i$.
To see if a key $x\in U$ is in $S$, we check if $x$ is in the list $L[h(x)]$. This takes time propotional to $1+|L[h(x)]|$.
We add one for the constant time to look up the list, and then add the number of elements in the list itself since we need to walk through them.

If $z \notin S$ and $h$ is universal, then the expected number of elements in $L[h(x)]$ is
\[
  E
\]