\subsection{Exact Exponential Algorithms \& Fixed-Parameter Tractable Problems}

\subsubsection{Exact Exponential Algorithms}
The $O^*$ notation is similar to the $O$ notation, except it will supress
runningtimes of polynomial itme. This means that factors that are not
exponential are supressed. For example $O(kn^kk^n) = O(n^kk^n)$ but we have
$O^*(kn^kk^n) = O^*(k^n)$.

Roughly speaking, parameterized complexity seeks the possibility of obtaining
algorithms whose running time can be bounded by a polynomial function of the
input length and, usually, an exponential function of the parameter.

% Parameterized Complexity add to our regular ``size'' or ``length'' dependent
% complexity measures, by also expressing the complexity in terms of an
% additional parameter. This parameter should be something the represents a
% factor in the algorithm which is not dependent on the input itself.

This section is basically about algotihms that solve problems in the NP group in
exponential time but give the correct result and not just an approximation.

\subsubsection{Fixed-Parameter Tractable Problems}

\textbf{CNF satisfiablility}
\begin{description}
\item[Parameter ``Clause Size''] The maximum snumber of $k$ literals a clause
  may contain. For $k = 2$ (2-CNF satisfiablility) the runningtime is polynomial
  time solvable, however for $k = 3$ (3-CNF satisfiablility) is NP-Complete.

\item[Parameter ``Number of Variables''] The number $n$ of different variables
  allowed in the formula. Since there is essentially $2^n$ different truth
  assignments, the problem can be solved in that number of steps, seeing that
  the result of each assignment can be calculated in a number of steps equal to
  the Number of clauses.

\item[Parameter ``Number of Clauses''] If the number of clauses i na formulae is
  bounded from above by $m$, the CNF problem can be solved in $1.24^m$ steps.

\item[Parameter ``Formula Length''] If the total length (counting the number of
  literal occurences in the formula) of the formula $F$ is bounded by above by
  $\ell = |F|$, then the problem can be solved in $1.08^\ell$ steps.
\end{description}