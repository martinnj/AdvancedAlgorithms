\subsection{Exact Exponential Algorithms \& Fixed-Parameter Tractable Problems}

\subsubsection{Exact Exponential Algorithms}
The $O^*$ notation is similar to the $O$ notation, except it will suppress
running-times of polynomial itme. This means that factors that are not
exponential are suppressed. For example $O(kn^kk^n) = O(n^kk^n)$ but we have
$O^*(kn^kk^n) = O^*(k^n)$.

\noindent \textbf{Size vs length:} When we talk about the running time we usually talk about
the time in relation to the input size of length. For instance: Given a graph, the input size
will be $O(V + E)$ while the length will be the number of bits it takes to encode the input with any reasonable encoding.

For the travelling salesman problem (er permutation problem) the input is a set of $n$ cities, where we want to find
the correct permutation. In this case candidate solutions are sets of $n$ cities, of which there is $n!$. Thus
the trivial algorithm runs in $O^*(n!)$.

\noindent \textbf{Travelling Salesman using Dynamic Programming} \\
For a subset of cities $S \subset \{1,2,...,n\}$ that includes $1$, and $j \in S$, let $C(S,j)$ be
the length of the shortest path visiting each node in $S$ exactly once, starting at $1$ and ending at $j$.
When $|S| > 1$, we define $C(S,1) = \infty$ since the path cannot start and end at $1$.

We can express $C(S,j)$ in smaller sub-problems. We start at $1$ and end at $j$; for the second to last
city we have to pick some $i \in S$, so the overall path length is the distance from $1$ to $i$; namely,
$C(S-\{j\},i)$ plus the length of the final edge $d_{ij}$. We pick the best such $i$:
\[
  C(S,j) = \underset{i\in S:i\neq j}{\text{min}} C(S-\{j\},i)+d_{ij}
\]




\subsubsection{Fixed-Parameter Tractable Problems}
Roughly speaking, parameterized complexity seeks the possibility of obtaining
algorithms whose running time can be bounded by a polynomial function of the
input length and, usually, an exponential function of a parameter which is independent of the input.

\textbf{CNF satisfiablility}
\begin{description}
\item[Parameter ``Clause Size''] The maximum snumber of $k$ literals a clause
  may contain. For $k = 2$ (2-CNF satisfiablility) the runningtime is polynomial
  time solvable, however for $k = 3$ (3-CNF satisfiablility) is NP-Complete.

\item[Parameter ``Number of Variables''] The number $n$ of different variables
  allowed in the formula. Since there is essentially $2^n$ different truth
  assignments, the problem can be solved in that number of steps, seeing that
  the result of each assignment can be calculated in a number of steps equal to
  the Number of clauses.

\item[Parameter ``Number of Clauses''] If the number of clauses i na formulae is
  bounded from above by $m$, the CNF problem can be solved in $1.24^m$ steps.

\item[Parameter ``Formula Length''] If the total length (counting the number of
  literal occurences in the formula) of the formula $F$ is bounded by above by
  $\ell = |F|$, then the problem can be solved in $1.08^\ell$ steps.
\end{description}

\todo[inline]{VRÆÆÆÆÆÆÆÆLLLLL :(}