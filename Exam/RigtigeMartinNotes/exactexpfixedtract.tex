\subsection{Exact Exponential Algorithms \& Fixed-Parameter Tractable Problems}

\subsubsection{Exact Exponential Algorithms}
The $O^*$ notation is similar to the $O$ notation, except it will suppress
running-times of polynomial itme. This means that factors that are not
exponential are suppressed. For example $O(kn^kk^n) = O(n^kk^n)$ but we have
$O^*(kn^kk^n) = O^*(k^n)$.

\noindent \textbf{Size vs length:} When we talk about the running time we usually talk about
the time in relation to the input size of length. For instance: Given a graph, the input size
will be $O(V + E)$ while the length will be the number of bits it takes to encode the input with any reasonable encoding.

For the travelling salesman problem (er permutation problem) the input is a set of $n$ cities, where we want to find
the correct permutation. In this case candidate solutions are sets of $n$ cities, of which there is $n!$. Thus
the trivial algorithm runs in $O^*(n!)$.

\noindent \textbf{Travelling Salesman using Dynamic Programming} \\
For a subset of cities $S \subset \{1,2,...,n\}$ that includes $1$, and $j \in S$, let $C(S,j)$ be
the length of the shortest path visiting each node in $S$ exactly once, starting at $1$ and ending at $j$.
When $|S| > 1$, we define $C(S,1) = \infty$ since the path cannot start and end at $1$.

We can express $C(S,j)$ in smaller sub-problems. We start at $1$ and end at $j$; for the second to last
city we have to pick some $i \in S$, so the overall path length is the distance from $1$ to $i$; namely,
$C(S-\{j\},i)$ plus the length of the final edge $d_{ij}$. We pick the best such $i$:
\[
  C(S,j) = \underset{i\in S:i\neq j}{\text{min}} C(S-\{j\},i)+d_{ij}
\]
The sub problems will be ordered by $|S|$.

\begin{codebox}
\Procname{$\proc{ExactTSP}(\{c_1,c_2,...c_n\},d)$}
\li $C(\{1\},1) \gets 0$
\li \For $s \gets 2$ \To $n$ \Do
\li   \For all subsets $S \subset \{c_1,c_2,...c_n\}$ of sizes $s$ and containing $1$ \Do
\li     $C(S,1) \gets \infty$
\li     \For all $j \in S, j\neq 1$ \Do
\li       $C(S,j) = \text{min}\{C(S-\{j\},i) + d_{ij} : i \in S, i \neq j\}$ \End\End\End
\li \Return $\text{min}_jC(\{1,...,n\},j) + d_{j1}$
\end{codebox}

There are at most $2^nn$ sub-problems, and each one takes linear time giving a final running time of $O(n^22^n)$ or $O^*(2^n)$.

\graphicc{1}{img/exacttsp.png}{A run of the exact TSP algorithm}{fig:exacttspsample}


\subsubsection{Fixed-Parameter Tractable Problems}
Roughly speaking, parameterized complexity seeks the possibility of obtaining
algorithms whose running time can be bounded by a polynomial function of the
input length and, usually, an exponential function of a parameter which is independent of the input.

We strive to understand a problem and its sub-problems in terms of parameters and their effects on the
running time, ideally the goal is to be able to form statements such as ``If some parameter $k$ is
small in problem $X$ then $X$ can be solved efficiently''. For instance in the vertex cover problem, we know that the solution
should be as few vertices as possible, so if $k$ is the solution size the exponential factor of the running time is less than $1.28^k$.

\noindent\textbf{CNF satisfiability}\\
A CNF problem is a conjunction of $m$ clauses where each clause consists of a disjunction of literals.
There there be $n$ different variables occurring in the formulae.

\begin{description}
\item[Parameter ``Clause Size''] The maximum number of $k$ literals a clause
  may contain. For $k = 2$ (2-CNF satisfiability) the running time is polynomial
  time solvable, however for $k = 3$ (3-CNF satisfiability) is NP-Complete.

\item[Parameter ``Number of Variables''] The number $n$ of different variables
  allowed in the formula. Since there is essentially $2^n$ different truth
  assignments, the problem can be solved in that number of steps, seeing that
  the result of each assignment can be calculated in a number of steps equal to
  the Number of clauses.

\item[Parameter ``Number of Clauses''] If the number of clauses in a formulae is
  bounded from above by $m$, the CNF problem can be solved in $1.24^m$ steps.

\item[Parameter ``Formula Length''] If the total length (counting the number of
  literal occurrences in the formula) of the formula $F$ is bounded by above by
  $\ell = |F|$, then the problem can be solved in $1.08^\ell$ steps.
\end{description}


\noindent \textbf{Maximum CNF Satisfiability}\\
An optimization version of the above problem. Here we want to satisfy as many clauses as possible and
not just the entire formulae. This problem shares the same parameters, but gives different effects.

\begin{description}
\item[Parameter ``Clause Size''] Even for $k = 2$ the problem is still NP-Complete.

\item[Parameter ``Number of Variables''] Like the decision problem, this one can be solved in $2^n$
  steps, where $n$ is the number of variables in the formulae. For $k=3$ it is still an open problem to obtain a better bound.
  Recently advances were made for $k=2$ allowing it to be solved in $1.74^n$ steps.

\item[Parameter ``Number of Clauses''] IFw e can bound the number of clauses above by $m$, the Maximum
Satisfiability problem can be solved in $1.33^m$ steps, and the Maximum 2 Satisfiability problem can be solved in $1.15^m$ steps.

\item[Parameter ``Formula Length''] If the total length (counting the number of
  literal occurrences in the formula) of the formula $F$ is bounded by above by
  $\ell$, Maximum Satisfiability problem can be solved in $1.11^\ell$ steps, and the Maximum 2 Satisfiability problem can be solved in $1.08^\ell$ steps.
\end{description}


\todo[inline]{VRÆÆÆÆÆÆÆÆLLLLL :(}