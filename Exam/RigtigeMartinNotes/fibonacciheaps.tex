\subsection{Fibonacci Heaps}
\todo[inline]{Write something about mergable heaps} Fibonacci heaps are a
datastructure that supports a set of operations qualifying it as a ``meargeable
heap'', meaning it supports the following operations.

\begin{itemize*}
  \item \texttt{Make-Heap()} Creates and returns a new heap with no elements.
  \item \texttt{Insert(H,x)} Inserts element $x$ whose key have already been
    filled in., into heap $H$.
  \item \texttt{Minimum(H)} Returns a pointer to the element in heap $H$ whose
    key is minimum.
  \item \texttt{Extract-Min(H)} Deletes the element from heap $H$ whose key is
    minimum, returning a pointer to the element.
  \item \texttt{Union(H$_1$,H$_2$)} Creates and returns a new heap that contains
    all the elements of both heaps. Both heaps are destroyed by the operation.
\end{itemize*}

Apart from the meargeable heap operations above, Fibonacci Heaps also support
the following two operations.

\begin{itemize*}
\item \texttt{Decrease-Key(H,x,k)} Assigns to element $x$ in heap $H$ the new
  key value $k$, which cannot be greater than it's current value.
  \item \texttt{Delete(H,x)} Deletes element $x$ from $H$.
\end{itemize*}

Fibonacci Heaps are by default min-heaps, but could just as well be max heaps,
then we would just replace the Minimum, Extract-Min and Decrease-Key operations
with Maximum, Extract-Max and Increase-Key instead.

Fibonacci Heaps have a benefit in the fact that many operations are run in
constat amortized time. So if these operations are used frequently, the
Fibonacci Heap is a well suited structure.

\begin{table}[h]
\begin{tabular}{lll}
  Procedure    & Binary heap(worst case) & Fibonacci heap (amortized) \\ \hline
  Make-heap    & $\Theta(1)$             & $\Theta(1)$ \\
  Insert       & $\Theta(lg n)$          & $\Theta(1)$ \\
  Minimum      & $\Theta(1)$             & $\Theta(1)$ \\
  Extract-Min  & $\Theta(lg n)$          & $O(lg n)$ \\
  Union        & $\Theta(n)$             & $\Theta(1)$ \\
  Decrease-Key & $\Theta(lg n)$          & $\Theta(1)$ \\
  Delete       & $\Theta(lg n)$          & $O(lg n)$
\end{tabular}
\caption{Amortized running times of normal binary heaps and Fibonacci Heaps.}
\end{table}

\subsubsection{Structure}

A Fibonaccci Heap is a collectoin of rooted trees that are ``min-heap
ordered''. That is, each tree obeys the minimum-heap property: The key of a node
is greater than or equal to the key of its parent. A node $x$ contains a pointer
$x.p$ to its parent and a pointer $x.child$ to any one of its children. This
list is called the child list. Each node also contains 2 pointers $x.left$ and
$x.right$, these points to a nodes siblings or to the node itself if it has n
osiblings. This forms a circular doubly-linked list called the child list. The
nodes may appear in the child list in any order.

Nodes have 2 additional properties, $x.degree$ which is how many children a node
have, and a boolean value $x.mark$. $x.mark$ indicates if $x$ has lost a child
since $x$ was made the child of another node. Nodes initially have $x.mark =$
False.

\graphicc{1}{img/fib_heaps_example}{An example of a Fibonacci Heap with(b) and
  without(a) the child-list.}{fig:figexample}

We access a fibonacci Heap by the pointer $H.min$ which points to the root of a
tree containing the minimum key, this node is called the minimum node. If there
are several nodes with the smallest key, any of those will work. When the heap
is empty, $H.min =$ NIL. The roots of all the tree in a Fibonacci Heap ar linked
togeter in a circular doubly-linked list called the ``root list''. $H.min$
points to a node in the root list. The heap also have one other property, $H.n$
which is the number of nodes currently in $H$.

\subsubsection{Operations}

\begin{description}
\item[Make-Heap] Simply creates a pointer $H.min =$ NIL; since there is no trees
  in $H$ at this point. This operation can be performed in $O(1)$ time.

\item[Insert] Insertion is done on constant time as well.
  \begin{codebox}
    \Procname{$\proc{Fib-Heap-Insert}(H,x)$}
    \li $x.degree \gets 1$
    \li $x.p \gets \const{nil}$
    \li $x.child \gets \const{nil}$
    \li $x.mark \gets \const{false}$
    \li \If $H.min == \const{false}$ \Do
    \li   Create a root list for $H$ containing just $x$.
    \li   $H.min = x$
    \li \Else insert $x$ into $H$'s root list
    \li   \If $x.key < H.min.key$ \Do
    \li     $H.min = x$
          \End
        \End
    \li $H.n = H.n + 1$

  \end{codebox}

\item[Minimum] Just follow the $H.min$ pointer and you're home safe.

\item[Extract-Min] This is the most complicated and expensive of the meargeable
  heap procedures, since it will be doing the work of actually consolidating the
  trees in the list. Most other procedures put of this work so that it can be
  done when using Extract-Min. This operation can be done in $O(\text{lg }n)$
  time.
  \begin{codebox}
    \Procname{$\proc{Fib-Heap-Extract-Min}(H)$}
    \li $z.min \gets H.min$
    \li \If $z \neq \const{nil}$ \Do
    \li   \For each child $x$ of $z$ \Do
    \li     add $x$ to the root list of $H$
    \li     $x.p \gets \const{nil}$ \End
    \li   remove $z$ from the root list of $H$
    \li   \If $z == z.right$ \Do
    \li     $H.min \gets \const{nil}$
    \li   \Else $H.min \gets z.right$
    \li     $\proc{Consolidate}(H)$ \End
    \li   $H.n = H.n - 1$ \End
    \li \Return $z$
  \end{codebox}

  Notice that $D(H.n)$ here will calculate the upper bound on the degree.
  \begin{codebox}
    \Procname{$\proc{Consolidate}(H)$}
    \li let $A[0..D(H.n)] $ be a new array
    \li \For $i \gets 0$ \To $D(H.n)$ \Do
    \li   $A[i] \gets \const{nil}$ \End
    \li \For each node $w$ in the root list of $H$ \Do
    \li   $x \gets w$
    \li   $d \gets x.degree$
    \li   \While $A[d] \neq \const{nil}$ \Do
    \li     $y \gets A[d]$ \Comment another node with the same degree as $x$
    \li     \If $x.key > y.key$ \Do
    \li       exchange $x$ with $y$ \End
    \li     $\proc{Fib-Heap-Link}(H,y,x)$
    \li     $A[d] \gets \const{nil}$
    \li     $d \gets d + 1$ \End
    \li   $A[d] \gets x$ \End
    \li $H.min \gets \const{nil}$
    \li \For $i \gets 0$ \To $D(H.n)$ \Do
    \li   \If $A[i] \neq \const{nil}$ \Do
    \li     \If $H.min == \const{nil}$ \Do
    \li       create a root list for $H$ containing just $A[i]$
    \li       $H.min \gets A[i]$
    \li     \Else insert $A[i]$ into $H$'s root list \Do
    \li       \If $A[i].key < H.min.key$ \Do
    \li         $H.min = A[i]$
  \end{codebox}

  \begin{codebox}
    \Procname{$\proc{Fib-Heap-Link}(H,y,x)$}
    \li remove $y$ from the root list of $H$
    \li make $y$ a child of $x$, incrementing $x.degree$
    \li $y.mark \gets \const{false}$
  \end{codebox}


\item[Union] Merging two fibonacci Heaps are done in constant:
  \begin{codebox}
    \Procname{$\proc{Fib-Heap-Union}(H_1,H_2)$}
    \li $H \gets \proc{Make-Fib-Heap}()$
    \li $H.min \gets H_1.min$
    \li Concatenate the root list oh $H_2$ with the root list of $H$.
    \li \If $(H_1.min == \const{nil})$ or ($H_2.min \neq \const{nil}$ and $H_2.min.key < H_1.min.key)$ \Do
    \li   $H.min \gets H_2.min$ \End
    \li $H.n \gets H_1.n + H_2.n$
    \li \Return $H$
  \end{codebox}

\item[DecreaseKey]
  We can decrease the key of any node using this method, it runs in $O(1)$ time.
  \begin{codebox}
    \Procname{$\proc{Fib-Heap-Decrease-Key}(H,x,k)$}
    \li \If $k > x.key$ \Do
    \li   \Error new key is greater than current key"\End
    \li $x.key \gets k$
    \li $y \gets x.p$
    \li \If $y \neq \const{nil}$ adn $x.key < y.key$ \Do
    \li   $\proc{Cut}(H,x,y)$
    \li   $\proc{Cascading-Cut}(H,y)$ \End
    \li \If $x.key < H.min.key$ \Do
    \li   $H.min \gets x$
  \end{codebox}
  \begin{codebox}
    \Procname{$\proc{Cut}(H,x,y)$}
    \li remove $x$ from the child list of $y$, decrementing $y.degree$
    \li add $x$ to the root list of $H$
    \li $x.p \gets \const{nil}$
    \li $x.mark \gets \const{false}$
  \end{codebox}
  \begin{codebox}
    \Procname{$\proc{Cascading-Cut}(H,y)$}
    \li $z \gets y.p$
    \li \If $z \neq \const{nil}$ \Do
    \li   \If $y.mark == \const{false}$ \Do
    \li     $y.mark \gets \const{true}$
    \li   \Else $\proc{Cut}(H,y,z)$
    \li     $\proc{Cascading-Cut}(H,z)$\End
  \end{codebox}


\item[Delete] Since it uses two previusly defined functions, the amortized
  runningtime is easily calculated to $O(\text{lg }n)$.
  \begin{codebox}
    \Procname{$\proc{Fib-Heap-Delete}(H,x)$}
    \li $\proc{Fib-Heap-Decrease-Key}(H,x,-\infty)$
    \li $\proc{Fib-Heap-Extract-Min}(H)$
  \end{codebox}
\end{description}

\subsubsection{Bounding Maximum Degree}
\todo[inline]{Starts on p. 523}
\begin{description}
  % \item[Lemma 19.1] Let $x$ be a node in a Fibonacci heap, and suppose that
  %   $x.degree = k$. Let $y_1, y_2,...y_k$ denote the children of $x$ in order
  %   of which they were linked to $x$, from earliest to latest. Then
  %   $y_1.degree \geq 0$ and $y_i.degree \geq i-2$ for $i = 2,3,...,k$.

% \item[Proof] Obviusly, $y_1.degree = \geq 0$. For $i \geq 2$, we note that
%   when $y_i$ was linked to $x$, all of $y_1,y_2,..,y_{i-1}$ were children of
%   $x$, and so we must have had $x.degree \geq i-1$.  Because $y_i$ is linked
%   to $x$ (by \texttt{Consolidate}) only if $x.degree = y_i.degree$, we must
%   also have $y_i.degree \geq i-1$ at that time. Since then, node $y_i$ has
%   lost at most one child, since it would have been cut from $x$ (by
%   \texttt{Cascading-Cut}) if it had lost two children. We conclude that
%   $y_i.degree \geq i-2$. \qed

%\item[Lemma 19.2]
%\item[Proof]

%\item[Lemma 19.3]
%\item[Proof]

  \item[Lemma 19.4] Let $x$ be any node in a Fibonacci heap, and let $k =
    x.degree$. Then size$(x) \geq F_{k+2} \geq \phi^k$, where $\phi =
    (1+\sqrt{5})/2$.
  \item[Proof]

  \item[Corollary 19.5] The maximum degree $D(n)$ of any node in an $n$-node
    Fibonacci heap is $O(\text{lg} n)$.
  \item[Proof] Let $x$ be any node in an $n$-node Fibonacci heap, and let $k =
    x.degree$. By Lemma 19.4, we have $n \geq \text{size}(x) \geq
    \phi^k$. Taking base-$\phi$ logarithms gives us $k \leq \text{log}_\phi n$.
    In fact, because $k$ is an integer, $k \leq \lfloor \text{log}_\phi n
    \rfloor$. The maximum degree $D(n)$of any node is thus $O(\text{lg}
    n)$. \qed

\end{description}


