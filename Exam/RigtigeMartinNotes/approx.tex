\subsection{Approximation Algorithms}

\subsubsection{Performance Ratios and Schemes}
An algorithm for a problem have an approximation ratio of $\rho(n)$ if, for any
input of size $n$, the cost $C$ of a solution produced by the algorithm is
within a factor of $\rho(n)$ of the cost $C^*$ of an optimal solution:
\[
  \text{max}\left( \frac{C}{C^*}, \frac{C^*}{C} \right) \leq \rho(n).
\]
If an algorithm achieves an approximation ratio of $\rho(n)$, we call it a
$\rho(n)$-approximation algoroithm. These notions apply to both
cost-minimization and cost-maximization problems.

For a miximization problem, $0 < C \leq C^*$, and the ratio $C^*/C$ gives the
factor by which the cost of an optimal solution is larger than the cost of the
approxmiation solution.

Similarly, for a minimization problem, $0 < C^* \leq C$, and the ratio $C/C^*$
gives the factor by which the cost of the approximate solution is larger than
the cost of an optimal solution.

Because we assume the costs are allways positive, the rations are always well
defined, the ratio of an algorithm is never less than 1, since $C/C^* \leq 1$
implies $C^*/C \geq 1$. Therefore a 1-approximation algorithm produce an optimal
solition and an approximation algorithm with a large approximation ratio may
return a solution far worse than optimal.


Below some schemes are outlined which allow us to give a value $\epsilon$ along
with the instance of the problem and achieve an approximation that have a
quality depending on the value.
\begin{description}
\item[Approximation Scheme] for an optimization problem is an approximation
  algorithm that takes as input, not only an instance of the problem, but also a
  value $\epsilon > 0$ such that for any fied $\epsilon$ the scheme is a
  $(1+\epsilon)$-approximation algorithm.

\item[Polynomial-Time Approximation Scheme] is an approximation scheme if for
  any fixed $\epsilon > 0$, the scheme runs in polynomia ltime in the size $n$
  of it's input instance. The running time of such a scheme can increase rapidly
  as $\epsilon$ decreases. For example, the running time of a polynomial-time
  approximation scheme might be $O(n^{(2/\epsilon)})$. Ideally, if $\epsilon$
  decrease by a constant factor, the running time to achieve the desired
  approximation should not incrase by more than a constant factor. (Not
  necessarily by the same factor $\epsilon$ was decreased with.)

\item[Fully Polynomial-Time Approximation Scheme] means the approximation
  algorithm runs in polynomial time in both $1/\epsilon$ and the size $n$ of the
  input instance. I.e. $O((1/\epsilon)^2n^3)$. With such a scheme, a constant
  factor decrease in $\epsilon$ comes with a constant factor increase in
  runningtime.
\end{description}


\subsubsection{Vertex Cover example}

A Vertex Cover of an undirected graph $G = (V,E)$ is a subset $V' \subseteq V$
such that if $(u,v)$is an edge of $G$, then either $u \in V'$ or $v \in V'$ or
both. The size of a vertex cover is the number of vertices in it.

The Vertex Cover Problem is to find a vertex cover of minimum size in a given
undirected graph, this is an optimization version of an NP-Complete decision
problem.

An approximation algorithm that will find a cover that is no more than twice the
size of the optimal cover is written here:
\begin{codebox}
\Procname{$\proc{Approx-Vertex-Cover}(G)$}
\li $C \gets \emptyset$
\li $E' \gets = G.E$
\li \While $E' \neq \emptyset$ \Do
\li   let $(u,v)$ be an arbitrary edge of $E'$
\li   $C = C \cup \{u,v\}$
\li   remove from $E$ every edge incident on either $u$ or $v$ \End
\li \Return $C$
\end{codebox}
The above algorithm runs in $O(V+E)$.

\begin{description}
\item[Theorem 35.1] \texttt{Approx-Vertex-Cover} is a polynomial-time
  2-approximation algorithm.
\item[Proof] It is already shown that the algorithm is a polynomial time
  algorithm.

  The set $C$ of vertices return must be a vertex cover since it loops until
  every edge in $G.E$ have been covered by some vertex.

  To see that the algorithm returns a vertex cover of at most twice the size of
  an optimal cover, let $A$ denote the set og edges that line 4 selects.  To
  cover all the edges in $A$, any vertex cover, in particular the optimal vertex
  cover $C^*$, must include at least one endpoint of each edge in $A$. No two
  edges share endpoints since when an edge is picked, all edges incident on its
  endpoints are removed. Thus no two edges in $A$ are covered by the same vertex
  from $C^*$, and we have the lower bound
  \[
    |C^*| \geq |A|
  \]
  on the size of an optimal vertex cover. Each execution of line 4 picks an edge
  which neither of its endpoints are already in $C$, yielding an upper boind on
  the size of the vertex cover returned
  \[
    |C| = 2|A|
  \]
  Combining the two bounds gives
  \begin{align*}
    |C| &= 2|A| \\
        &\leq 2|C*|
  \end{align*}
  thereby proving the theorem. \qed
\end{description}