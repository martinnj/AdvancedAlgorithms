\subsection{Max-Flow}
\subsubsection{Flow Network}
A flow network $G=(V,E)$ is a directed graph where each edge $(u,v) \in E$ have
a nonnegative capacity $c(u,v) \geq 0$. In addition, for any edge $(u,v)$ there
can be no antiparallel edge $(v,u)$.

Two vertices in the network have special characteristics the source $s$ and sink
$t$. We assume each vertex $v \in V$ lies on some path from $s$ to $t$, that is,
for each vertex $v \in V$, the flow network contains a path $s \leadsto v
\leadsto t$.

\subsubsection{Flow Definition}
We have a flow network $G = (V,E)$ with a source $s$ and a sink $t$, the network
has a capacity function $c(u,v)$. A flow is a real-valued function $f : V \times
V \rightarrow \mathbb{R}$ that satisfies the two following properties:
\begin{itemize}
  \item \textbf{Capacity Constraint}:\\
    For all $u,v \in V$, we require $0 \leq f(u,v) \leq c(u,v)$
  \item \textbf{Flow Conservation}:\\
    For all $u \in V - \{s,t\}$ we require
    \[
      \sum_{v\in V}f(v,u) = \sum_{v\in V}f(u,v)
    \]
    When $(u,v) \notin E$, there can be no flow from $u$ to $v$, and $f(u,v) =
    0$. We call the nonnegative quantity $f(u,v)$ the flow from vertex $u$ to
    vertex $v$. The value $|f|$ of a flow $f$ is defined as
    \[
      |f| = \sum_{v \in V} f(s,v) - \sum_{v \in V} f(v,s)
    \]
    that is the total flow out of the source minus the flow into the source.
\end{itemize}