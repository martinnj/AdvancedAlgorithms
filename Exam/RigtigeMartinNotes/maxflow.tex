\subsection{Max-Flow}
\subsubsection{Flow Network}
A flow network $G=(V,E)$ is a directed graph where each edge $(u,v) \in E$ have
a nonnegative capacity $c(u,v) \geq 0$. In addition, for any edge $(u,v)$ there
can be no antiparallel edge $(v,u)$.

Two vertices in the network have special characteristics the source $s$ and sink
$t$. We assume each vertex $v \in V$ lies on some path from $s$ to $t$, that is,
for each vertex $v \in V$, the flow network contains a path $s \leadsto v
\leadsto t$.

\subsubsection{Flow Definition}
We have a flow network $G = (V,E)$ with a source $s$ and a sink $t$, the network
has a capacity function $c(u,v)$. A flow is a real-valued function $f : V \times
V \rightarrow \mathbb{R}$ that satisfies the two following properties:
\begin{itemize}
  \item \textbf{Capacity Constraint}:\\
    For all $u,v \in V$, we require $0 \leq f(u,v) \leq c(u,v)$
  \item \textbf{Flow Conservation}:\\
    For all $u \in V - \{s,t\}$ we require
    \[
      \sum_{v\in V}f(v,u) = \sum_{v\in V}f(u,v)
    \]
    When $(u,v) \notin E$, there can be no flow from $u$ to $v$, and $f(u,v) =
    0$. We call the nonnegative quantity $f(u,v)$ the flow from vertex $u$ to
    vertex $v$.
\end{itemize}

\noindent The value $|f|$ of a flow $f$ is defined as
\begin{align*}
  |f| = \sum_{v \in V} f(s,v) - \sum_{v \in V} f(v,s)
\end{align*}
that is the total flow out of the source minus the flow into the source.  For an
example of a flow see Figure \ref{fig:flowexample}

\subsubsection{Antiparallel Edges and Multiple Sources/Sinks}
Since a flow network cannot contain anti-parallel edges, but we want to be able
to represent them in our graph, we need a way to do so. This is done by
inserting an additional node $v'$ and let one of the edges go through this node
instead, see Figure \ref{fig:antiedges} for an example.

If a network have multiple sources or sinks, we can convert it to a single
source/sink network by adding a supersource and supersink. An example of such
conversion can be seen in Figure \ref{fig:supersource}.

\subsubsection{Flow Examples}
\graphicc{1}{img/flow_example.png}{Example flow.}{fig:flowexample}
%%
\graphicc{1}{img/antiedges.png}{Conversion from antiparallel edges to proper
  flow.}{fig:antiedges}
%%
\graphicc{1}{img/supersource.png}{Example of a graph with multiple sources and
  sink, combined using a supersource and supersink.}{fig:supersource}

\subsubsection{Residual Networks}
Given a flow network $G$ and a flow $f$ the residual network $G_f$ consists of
edges and capacities that represent how we can change the flow on edges of $G$.
Suppose we have a flow network $G=(V,E)$ with source $s$ and sink $t$. Let $f$
be a flow in $G$, and consider a pair of vertices $u,v \in V$. We then define
the residual capacity $c_f(u,v)$ like this:
\[
  c_f(u,v) =
  \begin{cases}
    c(u,v) - f(u,v) & \text{if } (u,v) \in E, \\
    f(v,u)          & \text{if } (v,u) \in E, \\
    0               & \text{otherwise}.
  \end{cases}
\]
Given a flow network $G = (V,E)$ and a flow $f$, the residual network og $G$
induced by $f$ is $G_f(V,E_f)$, where
\[
  E_f = \{(u,v) \in V \times V : c_f(u,v) > 0\}
\]

An example of a residual network can be seen in Figure
\ref{fig:residualaugment}.

\graphicc{1}{img/residualaugment.png}{An example of a flow being augmented and
  showing the residual graph.}{fig:residualaugment}


\subsubsection{Augmenting Flows}
Augmenting paths are simply flows that can be added to other flows in order to
increase the flow value through the network. Augmenting flows are described
using the $\uparrow$ operator like so:
\begin{align*}
  (f\uparrow f')(u,v) =
  \begin{cases}
    f(u,v)+f'(u,v) - f'(v,u) & \text{if } (u,v) \in E, \\
    0                        & \text{otherwise}.
  \end{cases}
\end{align*}
an example of an applied augmenting path can bee seen in Figure
\ref{fig:residualaugment}.

\begin{description}
\item[Lemma 26.1] Let $G = (V,E)$ be a flow network with source $s$ and sink
  $t$, and let $f$ be a flow in $G$. Let $G_f$ be the residual network of $G$ be
  induced by $f$, and let $f'$ be a flow in $G_f$. Then the function $f\uparrow
  f'$ is a flow in $G$ with value $|f\uparrow f'| = |f| + |f'|$.
\end{description}

\subsubsection{Augmenting Paths}
Given a network $G=(V,E)$ and a flow $f$, an augmenting path is a simple path
from $s$ to $t$ in the residual network $G_f$. The shaded path in Figure
\ref{fig:residualaugment}(b) is an augmenting path. We can increase the flow on
each edge in the augmenting path $p$ by an amount equal to the residual capacity
of $p$ given by
\[
  c_f(p) = \text{min}\{c_f(u,v) : (u,v)\text{ is on }p\}
\]
That is, the smallest amount of spare capacity on any edge in $p$.

\begin{description}
\item[Lemma 26.2] let $G = (V,E)$ be a flownetwork, let $f$ be a flow in $G$,
  and let $p$ be an augmenting path in $G_f$. Define a function $f_p : V \times
  V \rightarrow \mathbb{R}$ by
  \[
    f_p(u,v) =
      \begin{cases}
        c_f(p) & \text{if } (u,v) \text{ is on } p, \\
        0      & \text{otherwise}.
      \end{cases}
  \]
  Then, $f_p$ is flow in $G_f$ with value $|f_p| = c_f(p) > 0$.

\item[Corollary 26.3] let $G = (V,E)$ be a flownetwork, let $f$ be a flow in
  $G$, and let $p$ be an augmenting path in $G_f$. Let $f_p$ be defined as in
  Lemma 26.2, and suppose that we augment $f$ by $f_p$. Then the function $f
  \uparrow f'$ is a flow in $G$ with value $|f\uparrow f'| = |f|+|f_p| > |f|$.
\end{description}

\subsubsection{Cuts in flow networks}
A cut $(S,T)$ of a flow network $G=(V,E)$ is a partition of $V$ into $S$ and $T
= V - S$ such that $s\in S$ and $t \in T$. If $f$ isa a flow, then the net flow
$f(S,T)$ across the cut $(S,T)$ is defined to be
\[
  f(S,T) = \sum_{u\in S}\sum_{v\in T} f(u,v) - \sum_{u\in S}\sum_{v\in T} f(v,u).
\]
The capacity of the cut $(S,T)$ is
\[
  c(S,T) = \sum_{u\in S}\sum_{v\in T} c(u,v).
\]

A \textbf{minimum cut} of a network is a cut whose capacity is minimum over all
cuts of the network. Note that for capacity we count only edges going from $S$
to $T$ while for flow we count edges going both directions.  Lemma 26.4 shows
that for a given flow $f$, the net flow across any cut is the same and it equals
$|f|$, the value of the flow.

\begin{description}
\item[Lemma 26.4] Let $f$ be a flow in a flow network $G = (V,E)$ with source
  $s$ and sink $t$, and let $(S,T)$ be any cut of $G$. Then the net flow across
  $(S,T)$ is $f(S,T) = |f|$

\item[Corollary 26.5] The value of any flow $f$ in a flow network $G$ is bounded
  from above by the capacity of any cut of $G$.
\end{description}

\noindent \textbf{Max-flow min-cut theorem}:
\begin{description}
\item[Theorem 26.6] If $f$ is a flow in a flow network $G = (V,E)$ with source
  $s$ and sink $t$, then the following conditions are equivalent:
  \begin{enumerate*}
    \item $f$ is a mximum flow in $G$.
    \item The residual network $G_f$ contains no augmenting paths.
    \item $|f| = c(S,T)$ for some cut $(S,T)$ of $G$.
  \end{enumerate*}

\item[Proof] $(1) \implies (2)$: If we assume $f$ is a maximum flow but there
  still is an augmenting path $p$ in the residual graph $G_f$, then by Corollary
  26.3, the flow found by $f\uparrow f_p$ is a flow with value strictly greater
  than |f|, contradicting the assumption that $f$ is a max flow.

  $(2) \implies (3)$: Suppose $G_f$ has no augmenting paths, that is, that $G_f$
  contains no paths from $s$ to $t$. We define
  \[
    S = \{v\in V : \text{there exists a path from $s$ to $v$ in $G_f$}\}
  \]
  and $T = V - S$. The partition $(S,T)$ is a cut: we have $s\in S$ trivially,
  and $t\notin S$ because there is no path from $s$ to $t$ in $G_f$.

  We now consider a pair of vertices $u\in S$ and $v \in T$. If $(u,v)\in E$, we
  must have $f(u,v) = c(u,v)$ otherwise $(u,v)\in E_f$ wich would place $v$ in
  $S$.

  \todo[inline]{Finish this off. See page 723-724}

  $(3) \implies (1)$:
  \todo[inline]{Finish this off. See page 723-724}
\end{description}


\subsubsection{Ford-Fulkerson}
\begin{codebox}
\Procname{$\proc{Ford-Fulerson-Method}(G,s,t)$}
\li initialize flow $f$ to 0
\li \While there exists an augmenting path $p$ in the residual network $G_f$
    \Do
\li      augment flow $f$ along $p$
    \End
\li \Return $f$
\end{codebox}