\subsection{Max-Flow}
\subsubsection{Flow Network}
A flow network $G=(V,E)$ is a directed graph where each edge $(u,v) \in E$ have
a nonnegative capacity $c(u,v) \geq 0$. In addition, for any edge $(u,v)$ there
can be no antiparallel edge $(v,u)$.

Two vertices in the network have special characteristics the source $s$ and sink
$t$. We assume each vertex $v \in V$ lies on some path from $s$ to $t$, that is,
for each vertex $v \in V$, the flow network contains a path $s \leadsto v
\leadsto t$.

\subsubsection{Flow Definition}
We have a flow network $G = (V,E)$ with a source $s$ and a sink $t$, the network
has a capacity function $c(u,v)$. A flow is a real-valued function $f : V \times
V \rightarrow \mathbb{R}$ that satisfies the two following properties:
\begin{itemize}
  \item \textbf{Capacity Constraint}:\\
    For all $u,v \in V$, we require $0 \leq f(u,v) \leq c(u,v)$
  \item \textbf{Flow Conservation}:\\
    For all $u \in V - \{s,t\}$ we require
    \[
      \sum_{v\in V}f(v,u) = \sum_{v\in V}f(u,v)
    \]
    When $(u,v) \notin E$, there can be no flow from $u$ to $v$, and $f(u,v) =
    0$. We call the nonnegative quantity $f(u,v)$ the flow from vertex $u$ to
    vertex $v$. The value $|f|$ of a flow $f$ is defined as
    \[
      |f| = \sum_{v \in V} f(s,v) - \sum_{v \in V} f(v,s)
    \]
    that is the total flow out of the source minus the flow into the source.
\end{itemize}

\subsubsection{Antiparallel Edges and Multiple Sources/Sinks}
Since a flow network cannot contain anti-parallel edges, but we want to be albe
to represent them in our graph, we need a way to do so. This is done by
inserting an additional node $v'$ and let one of the edges go through this node
instead, see Figure \ref{fig:antiedges} for an example.

If a network have multiple sources or sinks, we can convert it to a single
source/sink network by adding a supersource and supersink. An example of such
conversion can be seen in Figure \ref{fig:supersource}.

\subsubsection{Flow Examples}
\graphicc{1}{img/flow_example.png}{Example flow.}{fig:flowexample}
%%
\graphicc{1}{img/antiedges.png}{Conversion from antiparallel edges to proper
  flow.}{fig:antiedges}
%%
\graphicc{1}{img/supersource.png}{Example of a graphi with multiple sources and
  sink, combined using a supersource and supersink.}{fig:supersource}

\subsubsection{Residual Networks}
Given a flow network $G$ and a flow $f$ the residual network $G_f$ consists of
edges and capacities that represent how we can change the flow on edges of $G$.
Suppose we have a flow network $G=(V,E)$ with source $s$ and sink $t$. LEt $f$
be a flow in $G$, and consider a pair of vertices $u,v \in V$. We then define
the residual flow like this:
\[
  c_f(u,v) =
  \begin{cases}
    c(u,v) - f(u,v) & \text{if } (u,v) \in E, \\
    f(v,u)          & \text{if } (v,u) \in E, \\
    0               & \text{otherwise}.
  \end{cases}
\]

\graphicc{1}{img/residualaugment.png}{An example of a flow being augmented and
  showing the residual graph.}{fig:residualaugment}

An example of a residual network can be seen in Figure
\ref{fig:residualaugment}.


\subsubsection{Augmenting Paths}
Augmenting paths are simply flows that can be added to other flows in order to
increase the flow value through the network. Augmenting flows are described
using the $\uparrow$ operator like so:
\[
  (f\uparrow f')(u,v) =
  \begin{cases}
    f(v,u)+f'(u,v) - f'(v,u) & \text{if } (u,v) \in E, \\
    0                        & \text{otherwise}.
  \end{cases}
\]
an example of an applied augmenting path can bee seen in Figure
\ref{fig:residualaugment}.

\subsubsection{Ford-Fulkerson}
\begin{codebox}
\Procname{$\proc{Ford-Fulerson-Method}(G,s,t)$}
\li initialize flow $f$ to 0
\li \While there exists an augmenting path $p$ in the residual network $G_f$
    \Do
\li      augment flow $f$ along $p$
    \End
\li \Return $f$
\end{codebox}