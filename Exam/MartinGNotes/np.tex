\section{NP-Completeness: Disposition}
\begin{enumerate}
  \item P-Class, NP-Class, NPC-Class
  \item Showing problems to be NP-Complete
  \item Reductions and verifiability/certificates
  \item P vs NP vs NPC
\end{enumerate}

\section{NP-Completeness: Notes}
NP-Complete problems are a class of problems, whose status is ``unknown'', in that
no polynomial-time algorithm has yet been discovered for an NP-Complete problem, nor
has anyone yet been able to prove that such an algorithm can not exist. Therefore, the
so-called $P \neq NP$ problems, whose status is unknown. Several of the problems are tantalizing
because they seem quite similar to problems that we \textbf{can} solve in polynomial time. For example,
finding a single-source shortest path is solvable in polynomial time, finding a single-source longest
path is \textbf{not}. Finding an Euler tour (cycle that traverses each each exactly once, but is allowed
to visit vertices more than once) is solvable in polynomial time, a hamiltonian cycle (simple cycle that
contains each vertex) is not.

\subsection{P, NP, NPC Problems}
The class $P$ contains those problems that are solvable in polynomial time. More specifically, $P$ class
problems are solvable in $O(n^k)$ for some constant $k$, where $n$ is the size of the input.

The class $NP$ consists of those problems that are \textbf{verifiable} in polynomial time. That is, given
a ``certificate'' of a solution, we can verify that the certificate is correct in polynomial time. For example,
given a hamiltonian cycle $G = (V,E)$, a certificate could be a sequence of vertices $<v_1, v_2, v_3, \hdots, v_n>$
of $|V|$ vertices. We can easily check in polynomial time that $(v_i, v_{i+1}) \in E$ for $i = 1,2,3,\hdots,|V|-1$
and that $(v_{|V|},v_1) \in E$ as well.

Any problem in $P$ is also in $NP$, since if a problem is in $P$ then it is solvable in polynomial time. And so
we can get the solution directly in polynomial time. $P \subseteq NP$. The open question is whether or not $P$ is
a proper subset of $NP$ (meaning $P \subset NP$).

Informally, a problem is in the class $NPC$, and is called NP-Complete, if it is in NP and is at least as ``hard''
as any other problem in NP. If \textit{any} NP-Complete problem can be solved in polynomial time, then \textit{every}
NP-Complete problem can be solved in polynomial time.

\subsection{Showing problems to be NP-Complete}
When we want to show a problem to be NP-Complete, we make a statement about how hard the problem is (or how hard we
think it is), instead of how easy it is. We are trying to prove that no efficient algorithm likely exists. There are
\textbf{three} key concepts in showing a problem to be NP-Complete:
%
\subsubsection{Decision problems vs optimization problems}
Many problems are \textbf{optimization problems}, where the solution has an
associated legal value and we wish to find the best such value. For example,
for the shortest path problem, we are given an undirected graph $G$ and
vertices $u$ and $v$, and we wish to find a path from $u$ to $v$ that uses
fewest edges. \textbf{However}, NP-Completeness applies to \textbf{decision problems}
and not optimization problems, in which the answer is simply ``yes'' or ``no''.

However, it is often quite easy to convert an optimization problem into a decision problem, by
imposing a bound on the value we wish to optimize. For example: Given a directed graph $G$, vertices
$u$ and $v$ and an integer $k$, does a path exist from $u$ to $v$ consisting of at most $k$ edges?

Decision problems are in a sense ``easier'' than optimization problems, since if we can solve the optimization
problem then we can often trivially answer the decision problem. For example, if we find a path that solves the
optimization problem of shortest path, it is very easy to solve the decision problem posed above by simply checking
the length of the path. In other words, if the optimization problem is easy, then the decision problem is easy as well.
Conversely, if the decision problem is hard, then the related optimization problem is hard as well.

\subsubsection{Reductions}
The above applies even when both problems are decision problems. We call
the input to a particular problem an \textbf{instance}. For example, in PATH,
an instance would be a particular graph $G$, particular vertices $u$ and $v$
of $G$, and a particular integer $k$.

If we have a decision problem $A$ which we would like to solve in polynomial time, a decision
problem $B$ which we know we can solve in polynomial time and a transformation method that can
transform input from the form that $A$ accepts into the form that $B$ accepts, then we can solve
$A$ in polynomial time as follows:

First transform the input into a form acceptable for problem $B$ (polynomial time). Then run $B$ and
find an answer, in polynomial time. Then use that answer for $A$. Note that this assumes that when $B$
answers `Yes' then so does $A$, and when $B$ answers `No' then so does $A$.
If we have another decision problem $B$ which we know we can solve in polynomial time, which
