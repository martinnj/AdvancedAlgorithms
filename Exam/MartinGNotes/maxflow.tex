\section{Max-flow}
A flow network $G = (V,E)$ is a directed graph where each edge $(u,v) \in E$ has
a non-negative capacity $c(u,v) \geq 0$. If there is an edge $(u,v) \in E$ then
there is no edge $(v,u) \in E$. If $(u,v) \notin E$ then $c(u,v) = 0$ for convenience.
When $(u,v) \notin E$, $f(u,v) = 0$.

Flow networks have a source $s$ and a sink $t$. For each vertex $v \in V$, the flow
network contains a path $s \sim v \sim t$. The graph is therefore connected, meaning
$|E| \geq |V| - 1$.

A flow is a real-valued function $f : V \times V \rightarrow \mathbb{R}$ that satisfies
two properties:

\begin{description}
	\item[Capacity constraint:] For all $u,v \in V$, $0 \leq f(u,v) \leq c(u,v)$

	\item[Flow conservation:] For all $u \in V - \{s,t\}$, 
	$\sum_{v \in V} f(u,v) = \sum_{v \in V} f(v,u)$.
\end{description}

The value of a flow, $|f|$, is defined as:

\begin{align*}
	|f| = \sum_{v \in V} f(s,v) - \sum_{v \in V} f(v,s)
\end{align*}

In the \textbf{maximum-flow} problem, we are given a flow network $G$ and we wish to find
a maximum flow.

Edges are anti-parallel if there is both an edge $(u,v)$ and an edge $(v,u)$. This is not allowed,
and to get around this we instead introduce a new edge $x$ and re-structure the edges as follows:
$(u,x), (x,v), (v,u)$. The capacity of the new edges involving $x$ is the same as the capacity from
$(u,v)$. See page 711 in the book for an example.

\subsection{Multiple sources and sinks}
This can be accounted for by introducing a \textbf{supersink} and \textbf{supersource} with infinite
flow and capacity out to all of the sources and from all of the sinks to the supersink. See page 713.