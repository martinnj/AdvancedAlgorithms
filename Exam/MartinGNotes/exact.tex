\section{Exact Exponential Algorithms \& Fixed-parameter tractable problems: Disposition}
\begin{enumerate}
  \item $O^*$-notation
  \item Parameterized Complexity
  \item Travelling Salesman Problem using Dynamic Programming
  \item CNF-Satisfiablility.
\end{enumerate}

Formin \& Kratsch, Ch. 1
Fixed-parameter tractable problems - Selected notes.

\section{Exact Exponential Algorithms \& Fixed-parameter tractable problems: Notes}

\subsection{$O^*$-notation}
$\mathbb{O}^*$ notation is $\mathbb{O}$ notation where we suppress all
polynomially bounded factors. For example $O(kn^kk^n) = O(n^kk^n)$ but we have
$O^*(kn^kk^n) = O^*(k^n)$.

Different kinds of problems are measured based on different `kinds' of input:
\begin{itemize}
	\item Optimization problems on graphs are analyzed in terms of the number of vertices.
	\item Problems on sets are analyzed in terms of the number of elements.
	\item Problems on Boolean formulas are analyzed in terms of the number of variables.
\end{itemize}

There are often three broad categories of exact exponential algorithms:
\begin{itemize}
	\item Subset problems. Maximum independent set.
	\item Permutation problems. Travelling Salesman Problem.
	\item Partition problems. Graph coloring.
\end{itemize}

\subsection{Parameterized complexity}
\textit{Parameterized complexity} seeks to obtain algorithms whose running time can be bounded
by a polynomial function of the input length, and, usually, an exponential function of a parameter.
The parameter depends on the problem, e.g.\ the number of vertices in a graph, the number of variables
in a formula, etc. However, it is unclear whether parameterized complexity is directly applicable, as one
of the fundamental assumptions is that the parameter is independent of the input size.

\subsection{Travelling Salesman Problem}
Given cities $\{c_1,c_2,\hdots, c_n\}$ with distances $d(c_i,c_j)$, find the shortest path
going through all cities exactly once and returning to the starting point. Different point of view:
Find the permutation $\pi$ of $\{1 \hdots n\}$ that minimizes 
$\sum_{i=1}^{n-1} d(c_{\pi(i)}, c_{\pi(i+1)}) + d(c_{\pi(n), c_{\pi(1)}})$.

We use dynamic programming to compute for every pair $(S, c_i)$ the optimal tour of $S$, starting
at $c_1$ and ending at $c_i \rightarrow OPT[S,c_i]$. For $|S| > 1$, we have that
$OPT[S,c_i] = min\{OPT[S\setminus\{c_i\}, c_j] + d(c_j,c_i): c_j \in S\setminus \{c_i\}\}$. That is, the
optimal tour is the minimum distance from $c_1$ to $c_j$, plus the distance from $c_j$ to $c_i$,
for all subsets not containing $c_i$. For $|S|=k$, each step is $O(k^2)$, because we choose $k$ possible
$c_i$ and for each find the preceeding best $c_j$. The total number of subsets is
$\sum_{k=1}^{n-1} \binom{n}{k}k^2 = \mathbb{O}^*(2^n)$.

\subsection{CNF etc...}
\todo[inline]{FUUUUUDGE DET HER LAAAART!}