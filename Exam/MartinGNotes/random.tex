\section{Randomized Algorithms: Disposition}
\begin{enumerate}
	\item Randomized Quicksort
	\item Las Vegas, Monte Carlo
	\item 1-sided vs. 2-sided
	\item Bounding running times
	\item Markov \& Chebyshev's inequalities
	\item Randomized Selection
\end{enumerate}

Motwani \& Raghavan Ch. 1.1, 1.2, Ch. 3.2, 3.3

\section{Randomized Algorithms: Notes}

\subsection{Randomized quicksort}
In randomized quicksort, we wish to sort a set of numbers $S$ into ascending
order. We first select a pivot $y$ from $S$ at random. Subsequently, each
remaining element is compared to $y$ and put into $S_1$ if smaller, $S_2$
otherwise. We recursively sort $S_1$ and $S_2$ in the same manner, and then
output $S_1$ followed by $y$ followed by $S_2$.

The running time of randomized quicksort is analyzed in terms of the number of
comparisons made, as this dominates the running time of any (reasonably designed)
algorithm. In particular, we wish to find the \textit{expected} number of comparisons
made.

For $1 \leq i \leq n$, let $S_{(i)}$ denote the element of rank $i$ (the $i$th smallest element)
in the set $S$. Thus, $S_{(1)}$ denotes the smallest element and $S_{(n)}$ the largest. Define the
random variable $X_{ij}$ to assume the value $1$ if $S_{(i)}$ and $S_{(j)}$ are compared in an execution
and the value $0$ otherwise.

The total number of comparisons is then $\sum_{i=1}^n \sum_{j>i} X_{ij}$. The expected number of comparisons is
then $\mathbb{E}[\sum_{i=1}^n \sum_{j>i} X_{ij}] = \sum_{i=1}^n \sum_{j>i} \mathbb{E}[X_{ij}]$ by Linearity
of Expectation.

Let $p_{ij}$ denote the probability that $S_{(i)}$ and $S_{(j)}$ are compared in an execution. $X_{ij}$ can
assume the values $0$ and $1$, so:

\[
	\mathbb{E}[X_{ij}] = p_{ij} \times 1 + (1 - p_{ij}) \times 0 = p_{ij}
\]

To facilitate determining $p_{ij}$, we view the execution of randomized quicksort as a binary search tree $T$, where
each node is a distinct element of $S$. The root of the tree is the pivot $y$ we choose initially; the left sub-tree
contains elements in $S_1$ and the right sub-tree elements in $S_2$. The structure of the sub-trees is determined
recursively by subsequent runs of randomized quicksort. Importantly, the root of the tree ($y$) is compared to elements
in $S_1$ and $S_2$, but elements in $S_1$ are never compared to elements in $S_2$. Thus, there is a comparison between
an $S_{(i)}$ and $S_{(j)}$ only if one is the ancestor of the other.

This makes $T$ a random binary search tree. We are for this analysis interested in the level-order traversal of the
nodes of $T$. Such a traversal is a permutation $\pi$ obtained by visiting the nodes of $T$ in increasing order of
the level numbers, and in a left-to-right order within each level; recall that the $i$th level of the tree is the set
of all nodes at distance exactly $i$ from the root node.

To compute $p_{ij}$ we make use of two observations:
%
\begin{enumerate}
	\item There is a comparison between $S_{(i)}$ and $S_{(j)}$ if and only if $S_{(i)}$ or $S_{(j)}$ occur
	earlier in the permutation $\pi$ than any element $S_{(l)}$ such that $i < l < j$. Intuitively, they are
	only compared if there is some pivot node $l$ that puts $i$ in $S_1$ and $j$ in $S_2$. To see this, let
	$S_{(k)}$ be the earliest in $\pi$ from among all elements of rank between $i$ and $j$. If $k \notin \{i,j\}$
	then $S_{(i)}$ will belong to the left sub-tree of $S_{(k)}$ while $S_{(j)}$ will belong to the right sub-tree
	of $S_{(k)}$, implying that there is no comparison between $S_{(i)}$ and $S_{(j)}$. Conversely, if
	$k \in \{i,j\}$ then it must be that there is a parent-child relationship between $S_{(i)}$ and $S_{(j)}$
	implying that the two are compared.

	\item Any of the elements $S_{(i)}, S_{(i+1)}, \hdots, S_{(j)}$ is equally likely to be chosen as the pivot and
	hence appear first in $\pi$. Thus, the probability that the first element is either $S_{(i)}$ or $S_{(j)}$ is
	exactly $2 / (j - i + 1)$.
\end{enumerate}
%
This establishes that $p_{ij} = 2/(j - i + 1)$. The expected number of comparisons is thus:
%
\begin{align*}
	\sum_{i=1}^n \sum_{j>i} p_{ij} &= \sum_{i=1}^n \sum_{j>i} \frac{2}{j-i+1} \\
	&\leq \sum_{i=1}^n \sum_{k=1}^{n-i+1} \frac{2}{k} \\
	&\leq 2\sum_{i=1}^n \sum_{k=1}^{n} \frac{1}{k}
\end{align*}
%
It follows that the expected number of comparisons is bounded by $2nH_n$, where $H_n$ is the \textit{nth}
Harmonic number, defined by $H_n = \sum_{k=1}^n 1/k$. We know that $H_n \sim ln\ n + \Theta(1)$, making the
expected running time of randomized quicksort $O(n\ lg\ n)$

\subsection{Las Vegas \& Monte-Carlo}
An algorithm that \textit{always} gives the correct solution, but whose running time is variable or perhaps even
unbounded, is a \textbf{Las Vegas} algorithm. In contrast, an algorithm that may sometimes produce an incorrect
solution (although we can often bound the probability of this), but we know will always terminate, is a
\textbf{Monte-Carlo} algorithm. One useful property of a Monte-Carlo algorithm is that we can reduce the risk of
failure to be arbitrarily small by running the experiment repeatedly with
independent random choices each time.

For decision problems (yes or no problems), there are two kinds of Monte-Carlo
algorithms: those with one-sided error and those with two-sided error. A
Monte-Carlo algorithm is said to have a two-sided error if there is a non-zero
probability that it errs when it outputs either \texttt{YES} or \texttt{NO}. It is said
to have one-sided error if the probability that it errs is zero for at least one of the
possible outputs (\texttt{YES} / \texttt{NO}).

By definition, a Las Vegas algorithm is a Monte-Carlo algorithm with error
probability $0$. We say that a Las Vegas algorithm is an \textit{efficient}
Las Vegas algorithm if on any input its expected running time is bounded by a
polynomial function of the input size. Similarly, we say that a Monte-Carlo
algorithm is an \textit{efficient} Monte-Carlo algorithm if on any input its
worst-case running time is bounded by a polynomial function of the input size.

\subsubsection{Markov \& Chebyshev's inequalities}
\begin{description}
\item[Markov inequality] Let $Y$ be a random variable assuming only non-negative
  values. Then for all $t \in \mathbb{R}^+$,
  \[
    \text{Pr}[Y \geq t] \leq \frac{\text{E[Y]}}{t}.
  \]
  Equivalently,
  \[
    \text{Pr}[Y \geq k\text{E}[Y]] \leq \frac{1}{k}.
  \]
\item[Proof] Define a function $f(y)$:
  \[
   f(y) = \begin{cases}
     1 & \text{iff } y \geq t\\
     0 & \text{otherwise.}
   \end{cases}
  \]
  Then $\text{Pr}[Y \geq t] = \text{E}[f(y)]$. Since $f(y) \leq y/t$ for all $y$,
  \[
    \text{E}[f(Y)] \leq \text{E}\left [\frac{Y}{t} \right ] = \frac{\text{E}[Y]}{t},
  \]
  and the theorem follows. \qed
\end{description}

\begin{description}
\item[Chebyshevs inequality] Let $X$ be a random variable with expectation
  $\mu_X$ and a standard deviation of $\sigma_X$. Then for any $t \in
  \mathbb{R}^+$,
  \[
    \text{Pr}[|X - \mu_X| \geq t\sigma_X] \leq \frac{1}{t^2}
  \]
\item[Proof] Note that
  \[
    \text{Pr}[|X - \mu_X| \geq t\sigma_X] = \text{Pr}[(X - \mu_X)^2 \geq
    t^2\sigma_X^2]
  \]
  The random variable $Y = (X - \mu_X)^2$ has expectation $\sigma_X^2$, and
  applying the Markov inequality to $Y$ bounds this probability from above by
  $1/t^2$. \qed
\end{description}